\documentclass[10pt]{article}

%%%%%%%%%%%%%%%%%%%%%%%%%%%%%%%%%%%%%%%%%%%%%
% Package Inclusion and Document Formatting %
%%%%%%%%%%%%%%%%%%%%%%%%%%%%%%%%%%%%%%%%%%%%%
\usepackage{geometry,amsmath,amsthm,mathrsfs,amssymb,graphicx,bm,hyperref,url,pdfsync,fancyhdr}
\pagestyle{fancy}
\numberwithin{equation}{section}
%%%%%%%%%%%%%%%%%%%
% Custom Commands %
%%%%%%%%%%%%%%%%%%%
\newcommand{\n}{\noindent}
\newcommand{\norm}[1]{\left\lvert#1\right\rvert}
\newcommand{\avg}[1]{\left\langle#1\right\rangle}

%%%%%%%%%%%%%%%%%%%%%%%%%%
% Title Page Information %
%%%%%%%%%%%%%%%%%%%%%%%%%%

\title{Notes for PHYS 232: Stellar Structure}
\author{Bill Wolf}
\date{\today}

\begin{document}

\vfill\maketitle\vfill \newpage

\tableofcontents \newpage

%%%%%%%%%%%%%%%%%%%%%%
% January 9, 2012 %
%%%%%%%%%%%%%%%%%%%%%%

\section{Introduction}
	\emph{Monday, January 9, 2012}
	\subsection{The HR Diagram} 
	Most stars shine predominantly in the optical range of the electromagnetic spectrum. As a result, we get most of our information about stars by observing their optical output.  It makes sense, then, that we might organize stars by there color, which is indicative of their surface temperature. When plotting a population of stars' luminosities against their surface temperature, we note a strong correlation between the two. As it turns out, the controlling parameter for these quantities is the mass of the star, at least while the star is on the \textbf{Main Sequence} (stars burning hydrogen to helium). The correlation between the mass of a main-sequence star and its luminosity is incredibly strong (see HR diagram examples).
	\begin{figure}
		\centering
		\includegraphics[width=6in]{hr_local.pdf}
		\caption{An HR diagram for stars in the local neighborhood (shamelessly stolen from Google Images)}
		\label{HR.1f}
	\end{figure}
	\subsection{Conditions for a Star on the HR Diagram}
	We are interested in knowing what defines the regime where a star can reside in a particular $L,\,T_{\mathrm{eff}}=[L/(4\pi\sigma_{\mathrm{SB}}R^2)]^{1/4}$. Why, for example, is there a dynamical range in the luminosity spanning over six orders of magnitude, while only a range of a factor of about 5 in the effective temperature? To gain some perspective, we might observe the number of stars as a function of brightness. We organize these stars by their \textbf{spectral type} (a rough measure of how big the star is) and find their approximate \textbf{mass density} (the amount of mass contained in these stars per unit volume):
	\begin{center} 
	\begin{tabular}{l l}
		Spectral Type & $\rho\ (M_\odot/\mathrm{pc^3})$\\
		\hline
		O-B & 0.4\\
		A-F & 4\\
		G-M & 40\\
		WD's & 20
	\end{tabular}
	\end{center}
	Here we've used the standard labels for different spectral types, O, B, A, F, G, K, M, L, T, which are roughly in decreasing order of size and temperature (the reasoning for this scale is historical rather than logical, and the ordering is often remembered by the mnemonic, ``Oh Be A Fine Gal/Guy, Kiss Me! Less Talk!''). We see that the big, bright stars form an exceedingly small portion of the amount of stellar mass in our galaxy. We will find that this is because large stars exhaust their fuel much more quickly than smaller stars, and thus live and die much faster. We'll now observe another type of classification of stars used in our neighborhood, the Milky Way Galaxy.
		
	\subsubsection{Population I Stars}
	From Earth, the center of the galaxy is approximately 8.5 kpc away. Meanwhile, the disk is only about 100 pc wide.  We've observed that stars in the thin disk (commonly known as \textbf{Population I Stars}) are orbiting at the orbital velocity with only a small amount of axial and radial motion. They are essentially dynamically cold and in nearly circular orbits. This is indeed where most of the \textbf{interstellar medium} (ISM) resides, causing much of the star formation in the galaxy. This region is also very metal rich. That is, compared to other parts of the universe, there is a much higher concentration of elements heavier than helium present. We will denote the mass fraction of these ``metals'' with the letter $Z$, and in this region, we have $Z\sim 1-2\,\%$. These metals come from a previous generation of stars, who died in the past, giving off the metals we now have.	
	
	\subsubsection{Population II Stars}
	 \textbf{Population II Stars} reside mostly in the spheroid in the center of the galaxy. These are older stars in regions where star formation is largely shut down. Typically they are metal poor, with metallicities as low as $Z=10^{-4}Z_\odot$. Kinematically, they are often on radial orbits (rather than their more azimuthal population I counterparts in the disk). We typically say that the globular clusters are part of this population. Sometimes these stars are seen passing through the disk at velocities comparable to the orbital velocities, and are easily identified due to their high velocities and unique spectra (due to the low metallicities).

%%%%%

\section{A Simple Hydrostatic Atmosphere}
	 \subsection{Scale Height and Column Depth}
	 Before tackling the physics of stars, we first consider a simple toy model-- the isothermal, plane parallel atmosphere. This model is somewhat applicable to the thin stellar atmosphere near the surface of a star, where curvature can be neglected and the acceleration due to gravity is nearly uniform.\\
	 
	 \n Consider an atmosphere where the local acceleration due to gravity, $\mathbf{g}$ is constant in value and direction. The atmosphere is composed of an isothermal ideal gas with temperature $T$. We wish to find the distribution of particles in this atmosphere.\\
	 
	 \n In a strictly statistical sense, we would expect the energy distribution to be comparable to $e^{-E/kT}$ (recall that the atmosphere is isothermal, so the average kinetic energy is uniform throughout). In our case, the energy of particles is linear in height, so we expect this probability to be proportional to $e^{-mgh/kT}$.\\
	 
	 \n We will let $m_B\approx m_p$ be the baryon mass, $\mu$ be the mean molecular mass (measured in AMU), and $\rho$ is the density in $\mathrm{g\,cm^{-3}}$. We suppose that the gas is in hydrostatic balance, so we have
	 \begin{equation}
	 	\label{ippa.1} \frac{dP}{dz}=-\rho g
	 \end{equation}
	 Combining this with the ideal gas law,
	 \begin{equation}
	 	\label{ippa.2} P=nkT
	 \end{equation}
	We find that
	\begin{equation}
		\label{ippa.3} kT\frac{dn}{dz}=-m_p\mu ng
	\end{equation}
	which in turn gives us
	\begin{equation}
		\label{ippa.4} \frac{d\ln n}{dz}=-\frac{m_p\mu g}{kT}
	\end{equation}
	Solving this differential equation gives the expected result
	\begin{equation}
		\label{ippa.5} n(z)=n(0)\exp\left(-\frac{m_p\mu gz}{kT}\right)=n(0)\exp\left(-\frac{z}{h}\right)
	\end{equation}
	where we have defined the \textbf{scale height} $h\equiv kT/(\mu m_pg)$, which is the e-folding distance in number density. As it turns out, the scale height for earth's atmosphere is approximately 10 km. This model is really only valid in cases where $h\ll R$, (where $R$ is the size of the object), so we now investigate the ratio of these two quantities:
	\begin{equation}
		\label{ippa.6} \frac{h}{R}=\frac{kT}{\mu m_p\frac{GM}{R^2}R}=\frac{1}{\mu}\frac{kT/m_p}{GM/R}\sim \frac{v_{\mathrm{th}}^2}{v_{\mathrm{esc}}^2}\ll 1
	\end{equation}
	So if this approximation is valid, the thermal velocities of particles are typically much smaller than the escape velocity of the central body, so a star could retain its own atmosphere (thankfully, Earth does the same to its atmosphere!) For stars, we will find that $kT_c/m_p\sim GM/R$. Then, \eqref{ippa.6} tells us that
	\begin{equation}
		\label{ippa.7} \frac{h}{R}\sim \frac{T_{\mathrm{eff}}}{T_c}.
	\end{equation}
	This tells us that stars are quite sharp-edged (their scale heights are very small compared to their radii). We can also deduce a physical meaning for the scale height as being how far a particle needs to fall to gain an energy comparable to $kT$.\\
	
	\n From the ideal gas law, we can easily see that the pressure will also fall off exponentially in this isothermal atmosphere. However, let's explore the pressure a bit more. First, we return to the ideal gas law,
	\begin{equation}
		\label{ippa.8} P=\frac{\rho}{\mu m_p}kT=nkT
	\end{equation}
	and the condition for hydrostatic equilibrium,
	\begin{equation}
		\label{ippa.9} dP=-\rho g\,dz.
	\end{equation}
	We now integrate \eqref{ippa.9} from $z=z$ to $z\to \infty$:
	\begin{align}
		\label{ippa.10} P(\infty)-P(z)&=-\int_z^\infty \rho(z') g\,dz'\\
		\label{ippa.11} P(z) &= g\int_z^\infty \rho(z')\,dz
	\end{align}
	Note that it is okay to take the integral to infinity so long as we are dealing with a constant $\mathbf{g}$. This result suggests the definition of the \textbf{column density}:
	\begin{equation}
		\label{ippa.12} y(z)\equiv \int_z^\infty \rho(z)\,dz
	\end{equation}
	On the surface of the earth, the column density is approximately $y=1000\,\mathrm{g\,cm^{-2}}$. Think of it as the amount of mass sitting above you per unit area at a given altitude. The column density is an important number (for us) to determine the details of heat transport. For now though, we can write the pressure in this isothermal atmosphere in a compact form: $P(z)=gy(z)$.
	\subsection{Mean Molecular Weights}
	We'll now make a useful definition for calculating pressures and other useful quantities. For an ideal gas, the total pressure of a mixed gas is simply
	\begin{equation}
		\label{mmw.1} P=\sum_{i=1}^N n_ikT
	\end{equation}
	where $n_i$ are just the number densities for each ion. The number density is computed via
	\begin{equation}
		\label{mmw.2} n_i=\frac{X_i\rho}{A_im_p}.
	\end{equation}
	where $X_i$ is the mass fraction of the $i^{\mathrm{th}}$ ion with mass number $A_i$ and $\rho$ is the overall mass density. Then the ion pressure is given by (assuming total ionization)
	\begin{equation}
		\label{mmw.3} P_{\mathrm{ion}}=kT\sum\frac{X_i\rho}{A_im_p}=\frac{kT\rho}{m_p}\sum\frac{X_i}{A_i}=\frac{kT\rho}{\mu_{\mathrm{ion}}m_p}.
	\end{equation}
	Where we have defined the \textbf{mean molecular weight} of the ions to be
	\begin{equation}
		\label{mmw.3a} \mu_{\mathrm{ion}}=\left[\sum\frac{X_i}{A_i}\right]^{-1}
	\end{equation}
	For the electrons, we have (assuming total ionization)
	\begin{equation}
		\label{mmw.4} P_e=n_ekT=kT\left(\sum Z_in_i\right)=\frac{kT\rho}{m_p}\sum\frac{Z_iX_i}{A_i}.
	\end{equation}
	(here $Z_i$ is the atomic number, not the metallicity). Then the total pressure is just the sum of these two,
	\begin{equation}
		\label{mmw.5} P=P_{\mathrm{ion}}+P_e=\frac{\rho kT}{m_p}\left(\frac{1}{\mu_e}+\frac{1}{\mu_i}\right)
	\end{equation}
	So we define the overall mean molecular weight via
	\begin{equation}
		\frac{1}{\mu}\equiv \frac{1}{\mu_e}+\frac{1}{\mu_i}
	\end{equation}
	One might think of this as the average weight of a particle that supplies pressure within a gas. Later, we'll see that this quantity, and its evolution, plays a large and critical role in the the nature of stellar evolution. Since fusion tends to decrease the pressure support, the star must continuously readjust its structure so as to hold itself up.\\
	
%%%%%%%%%%%%%%%%%%%%%%
%  January 11, 2012  %
%%%%%%%%%%%%%%%%%%%%%%
	\n \textit{Wednesday, January 11, 2012}
	\subsection{Electric Fields in Stars}
	Imagine a pure, ionized hydrogen atmosphere which is, on the large scale, electrically neutral. We wish to find the scale height in a plasma of ionized hydrogen. In this plasma, we have $n_p=n_e$ due to electric neutrality. Then the overall pressure in this hydrogen plasma is
	\begin{equation}
		\label{efs.2} P=2n_pkT
	\end{equation}
	Using hydrostatic equilibrium, we get
	\begin{equation}
		\label{efs.3} 2kT\frac{dn_p}{dz}=-m_pn_pg
	\end{equation}
	which in turn gives us the differential equation
	\begin{equation}
		\label{efs.4} \frac{d\ln n_p}{dz}=-\frac{m_pg}{2kT}
	\end{equation}
	Which gives us a scale height of
	\begin{equation}
		\label{efs.5} h=\frac{2kT}{m_pg}
	\end{equation}
	We need to look at both plasmas separately while incorporating the electric field created between the protons and electrons. For electrons, we have
	\begin{equation}
		\label{efs.6} \frac{1}{n_e}\frac{dP_e}{dz}=-m_eg-eE.
	\end{equation}
	Likewise for the protons,
	\begin{equation}
		\label{efs.7} \frac{1}{n_p}\frac{dP_p}{dz}=-m_pg+eE
	\end{equation}
	where we've assumed that the electric field points up (the protons are heavier and would thus sink below the electrons). Now adding \eqref{efs.6} and \eqref{efs.7}, we recover hydrostatic balance. However, subtracting the two equations will get us the electric field:
	\begin{equation}
		\label{efs.8} 0=-m_eg+m_pg-2eE,
	\end{equation}
	giving the result,
	\begin{equation}
		\label{efs.9} eE=\frac{1}{2}\left(m_p-m_e\right)g \quad \textrm{or}\quad e\mathbf{E}\approx -\frac{m_p\mathbf{g}}{2}.
	\end{equation}
	So this field does not directly affect hydrostatic balance, but it does dramatically impact the relative difficulty of (?UNREADABLE TEXT IN LECTURE NOTES?) in a white dwarf.
\section{Self-Gravitating Objects}
	So far we have only considered systems where the acceleration due to gravity is constant. In any self-gravitating object, this is obviously not true. We will, however, continue to assume that such objects do not rotate. Additionally, we will be ignoring mass loss. Essentially all we must write down are equations of mass conservation, momentum conservation, and energy conservation. We'll start with momentum conservation.
	\subsection{Momentum Conservation and the Free-Fall Timescale}
	The momentum equation for a fluid is just
	\begin{equation}
		\label{mc.1} \rho\frac{d\mathbf{v}}{dt}=\rho\mathbf{g}-\bm{\nabla}P
	\end{equation}
	This equation essentially states that a self-gravitating object is neither collapsing nor expanding. If we were to ``shut off'' gravity or the pressure gradient, the star would either explode or collapse, respectively. Such a collapse would occur on the \textbf{free-fall timescale}, which we will now derive. Taking the pressure gradient out of \eqref{mc.1}, we retrieve
	\begin{equation}
		\label{mc.2} \mathbf{g}=-\frac{Gm(r)}{r^2}\hat{r}
	\end{equation}
	For this derivation, we will be using a \textbf{Lagrangian coordinate system}. This is a system where the coordinates follow a particular fluid element. In essence, we are making the substitution
	\begin{equation}
		\label{mc.3} \frac{d}{dt}\to\frac{\partial}{\partial t}+\mathbf{v}\cdot\bm{\nabla}
	\end{equation}
	Returning back to the derivation, \eqref{mc.2} gives us
	\begin{equation}
		\label{mc.4} \frac{dv_r}{dt}=-\frac{Gm(r)}{r^2}
	\end{equation}
	Initially, we have $t=0$, $v_r=0$, and $r=r_0$ with the radial velocity given by $v_r=dr/dt$. Then our differential equation is
	\begin{equation}
		\label{mc.5} \frac{d^2r}{dt^2}=-\frac{Gm(r)}{r^2}
	\end{equation}
	As an order of magnitude estimate, this gives us
	\begin{equation}
		\label{mc.6}\frac{r}{t_{\mathrm{ff}}^2}\sim\frac{Gm}{r^2}\quad \Rightarrow \quad t_{\mathrm{ff}}^2\sim\frac{1}{Gm/r^3}
	\end{equation}
	So we define the free-fall timescale to be
	\begin{equation}
		\label{mc.7} t_{\mathrm{ff}}=\frac{1}{\sqrt{G\avg{\rho}}}
	\end{equation}
	This is also the same as the Keplerian orbital period, modulo some uninteresting constants. The punchline of this whole argument is that a star that is \emph{not} in hydrostatic balance will respond on a timescale of the free-fall timescale. From this alone, we may conclude that the sun (and all other stars not currently exploding) is in hydrostatic balance. We will then assume that all stars are always in hydrostatic balance.\\
	
	\subsection{The Virial Theorem}
	Stars are held up by gas pressure, radiation pressure, or both. The pressure gradients are what will be the ``restoring forces'' against gravity for our cases. In spherical symmetry, hydrostatic balance tells us
	\begin{equation}
		\label{rss.1} \frac{dP}{dr}=-\rho\frac{Gm(r)}{r^2}
	\end{equation}
	We will use this to derive the \textbf{Virial Theorem}, which relates the potential energy to the kinetic energy of a system. The equation of mass conservation states that
	\begin{equation}
		\label{rss.2} dm=4\pi r^2\rho(r)\,dr
	\end{equation}
	Now we multiply both sides of \eqref{rss.1} by $4\pi r^3\,dr$:
	\begin{align}
		\label{rss.3} \int 4\pi r^3\,dP&=-\int \rho\frac{Gm(r)}{r^2}4\pi r^3\,dr\\
		\label{rss.4} &= -\int\frac{Gm(r)dm}{r} = E_{\mathrm{GR}}
	\end{align}
	where $E_{\mathrm{GR}}$ is the gravitational binding energy. Performing a similar analysis to the left side of \eqref{rss.3} gives
	\begin{align}
		\label{rss.5} \int 4\pi r^3\,dr\frac{dP}{dr} &= \left.4\pi r^2P\right|_{0}^R-3\left[4\pi\int Pr^2\,dr\right]\\
		\label{rss.6} &= -3\int P4\pi  r^2\,dr\\
		\label{rss.7}&=-3\avg{P}V
	\end{align}
	where we've defined the average pressure to be the pressure
        averaged over volume. Also, in \eqref{rss.6} we've noted that
        the boundary terms must vanish since $P(R)=0$. Then the virial
        theorem tells us that
	\begin{equation}
		\label{rss.8}\boxed{\avg{P}=-\frac{1}{3}\frac{E_{\mathrm{GR}}}{V}}
	\end{equation}
	Now we examine the total energy:
	\begin{equation}
		\label{rss.9} E_{\mathrm{tot}}=E_{\mathrm{GR}}+E_{\mathrm{KE}}=-3\avg{P}V+E_{\mathrm{KE}}
	\end{equation}
	We need only relate the kinetic energy to the pressure to finish this equation off. For an ideal gas, we know that $P=NkT/V$, so the kinetic energy is $E_{\mathrm{KE}}=\frac{3}{2}NkT=\frac{3}{2}PV$. This gives a total energy of
	\begin{equation}
		\label{rss.10} E_{\mathrm{tot}}=-3\avg{P}V+\frac{3}{2}\avg{P}V=-E_{\mathrm{KE}}
	\end{equation}
	Interestingly, this requires a negative heat capacity. That is, an increase in the temperature of the system causes a net \emph{decrease} in total energy. However for radiation, pressure is given by $P=\frac{1}{3}aT^4$ and $E/V=aT^4$. Taking this to its conclusion gives us
	\begin{equation}
		\label{rss.11} E_{\mathrm{tot}}\to 0\ \textrm{as the particles become relativistic}
	\end{equation}
	The origin of this result is in the momentum-energy relation of relativistic particles and non-relativistic particles. That is, $E=pc$ for ultra-relativistic particles and $E=p^2/2m$ for non-relativistic particles.\\
	
	\n The limiting energy of ultra-relativistic stars puts an upper level on the mass of large stars, since a total energy of a star being zero means unbinding the star. In the ``normal case'' of an ideal gas star, the more traditional form of the virial theorem applies:
	\begin{equation}
		\label{rss.12} \frac{E_{\mathrm{KE}}}{\mathrm{mass}}\sim\frac{GM}{R}
	\end{equation}
	This is why stars typically behave with a negative heat capacity. That is, as a star radiates, $E_{\mathrm{tot}}$ is more negative, meaning that $R$ must decrease and the temperature $T$ (essentially the kinetic energy per particle) rises. This behavior would have to continue until a new energy source became available.
	\subsection{Applications of the Virial Theorm}
	The gravitational energy of an object is typically given by
	\begin{equation}
		\label{avt.1}E_{\mathrm{GR}}\approx -\frac{GM^2}{R}
	\end{equation}
	Using the virial theorem, we have
	\begin{equation}
		\label{avt.2} -E_{\mathrm{GR}}=3\avg{P}V=3Nk\avg{T}
	\end{equation}
	Or,
	\begin{equation}
		\label{avt.3} \frac{GM}{R}\left(Nm_p\right)\sim 3NkT
	\end{equation}
	So we have
	\begin{equation}
		\label{avt.4} \boxed{kT\sim \frac{GMm_p}{R}}
	\end{equation}
	This temperature is the temperature of most of the material and is $T\sim T_c\sim \mathrm{core}$. For the sun, we then have $T\sim 10^7\ \mathrm{K}$. Interestingly, assuming hydrostatic equilibrium was all we needed to get a rough estimate of the sun's core temperature! One might note, though, that the surface temperature is significantly lower than the core temperature, so we must assume that there is heat loss taking place in the sun. Today the luminosity of the sun is
	\begin{equation}
		\label{avt.5} L_\odot = 4\times 10^{33}\ \mathrm{erg/s}
	\end{equation}
	If we assume there is no energy source for the sun other than gravitational energy, we can come up with a timescale (called the \textbf{Kelvin-Helmholtz timescale})
	\begin{equation}
		\label{avt.6} t_{\mathrm{KH}}=\norm{\frac{E_{\mathrm{GR}}}{L}}\approx 3\times 10^7\ \mathrm{years}
	\end{equation}
	for the sun. This has been known for awhile and since the Earth is known to have existed much longer than $t_{\mathrm{KH}}$, scientists deduced that another energy source within the sun was needed to explain its longevity. We now know that this energy source is, of course, fusion. Note that at the center of the sun, the temperature of $10^7\ \mathrm{K}$ corresponds to an energy per particle of about 1 keV. The binding energy of helium is approximately 7MeV, approximately 7000 times bigger than the thermal content. Thus, the sun could last approximately 7000 times longer, bringing the projected lifetime of the sun up to a more reasonable (but still wrong) number of about 200 billion years. We conclude that nuclear energy is a more promising form of energy for the sun than chemical energy.\\
	
\n\textit{Wednesday, January 18, 2012}
	\subsection{Gas Pressure and Radiation Pressure}
		Recall from the case of the constant density star that the gravitational energy is given by
		\begin{equation}
			\label{GPRP.1} E_{\mathrm{GR}}=-\frac{3}{5}\frac{GM^2}{R}
		\end{equation}
		And the average pressure is given by the virial theorem to be
		\begin{equation}
			\label{GPRP.2} \avg{P}=-\frac{1}{3}\frac{E_{\mathrm{GR}}}{V}=(n_e+n_p)kT=2n_pkT=\frac{2\rho kT}{m_p}
		\end{equation}
		Here we've sort of assumed that the star is
                isothermal. This tells us that the average thermal
                energy is given by
		\begin{equation}
			\label{GPRP.3} kT=\frac{1}{10}\frac{GMm_p}{R}
		\end{equation}
		Where the mass is given by
		\begin{equation}
			\label{GPRP.4} M=\rho\frac{4\pi}{3}R^3
		\end{equation}
		and the central temperature is given approximately by (scaled by solar units)
		\begin{equation}
			\label{GPRP.5} T_c\approx2\times 10^6\,\mathrm{K}\left(\frac{\rho_c}{1\,\mathrm{g\,cm^{-3}}}\right)^{1/3}\left(\frac{M}{M_\odot}\right)^{2/3}
		\end{equation}
		These scalings are actually recovered in simulations
                (see MESA plot from class). Here we've only considered
                the case of the pressure due to an ideal gas, thus far
                ignoring the contributions from radiation pressure. We
                then want to know when radiation pressure becomes
                comparable to gas pressure. That is,
		\begin{equation}
			\label{GPRP.6} P_{\mathrm{rad}}=\frac{1}{3} aT^4\geq P_{\mathrm{gas}}
		\end{equation}
		The temperature in the star is approximately
		\begin{equation}
			\label{GPRP.7} kT\sim\frac{GMm_p}{R}
		\end{equation}
		and the pressure gradient is, (again, very approximately)
		\begin{equation}
			\label{GPRP.8} \frac{dP}{dr}=-\rho g\approx \frac{P_c}{R}\sim\frac{M}{R^3}\frac{GM}{R^2}
		\end{equation}
		Then the condition we are seeking is
		\begin{equation}
			\label{GPRP.9} \frac{1}{3}a\left(\frac{GMm_p}{Rk}\right)^4\gtrsim\frac{GM^2}{R^4}
		\end{equation}
		Interestingly, $R$ cancels in \eqref{GPRP.9}, so this condition is dependent only on the mass of the star. Thus, we can get a hard limit that is independent of any other properties of the star. Dropping tons more constants, this gives
		\begin{equation}
			\label{GPRP.10} M^2>\frac{k^4}{a G^3m_p^4}
		\end{equation}
		Recall that the radiation constant is given by $a=\frac{\pi^2}{15}\frac{k^4}{(\hbar c)^3}$. Putting this in \eqref{GPRP.10}, we have
		\begin{equation}
			\label{GPRP.11} \frac{M^2}{m_p^2}>\frac{k^4(\hbar c)^3}{G^3m_p^6 k^4}\sim\left(\frac{\hbar c}{G m_p^2}\right)^3
		\end{equation}
		Then the limit on the mass is then
		\begin{equation}
			\label{GPRP.12}\boxed{ M>m_p\left(\frac{\hbar c}{Gm_p^2}\right)^{3/2}}
		\end{equation}
		stars above this mass (approximately) have significant radiation pressure. Recall the fine structure constant
		\begin{equation}
			\label{GPRP.13} \alpha=\frac{1}{137} =\frac{e^2}{\hbar c}
		\end{equation}
		Noting that the Coulomb energy is
		\begin{equation}
			\label{GPRP.14} E_{\mathrm{Coulomb}}=\frac{e^2}{r}
		\end{equation}
		Analagous to \eqref{GPRP.14}, we define a
                dimensionless measure of the strength of gravity,
                which appears in \eqref{GPRP.12}:
		\begin{equation}
			\label{GPRP.15} \alpha_\mathrm{G}=\frac{Gm_p^2}{\hbar c}\approx 6\times 10^{-39}
		\end{equation}
		Then the fundamental stellar mass given in \eqref{GPRP.12} is
		\begin{equation}
			\label{GPRP.16} M>m_p\frac{1}{\alpha_{\mathrm{G}}^{3/2}}\approx 2M_\odot
		\end{equation}
		After all that work\ldots it turns out that the mass where radiation pressure \emph{actually} starts to matter is closer to $60-90\ M_\odot$.\\
		
		\n Of interest in this case is that as $P_{\mathrm{rad}}\gg P_{\mathrm{gas}}$, then $E_{\mathrm{tot}}\to 0$ from the virial theorem. In this state, the star has enough energy to unbind itself, so radiation pressure sets an upper limit on the mass of a star.
	\subsection{Summary}
		Note that here we have used hydrostatic balance to
                find the central temperature as a function of mass and
                radius. Additionally we have realized that energy
                losses from the surface require the radius of a star
                to decrease and the core temperature to increase (at
                least until another energy source is present). We will
                later show that the main sequence is just that place
                where the power generated by nuclear reactions is
                equal to that released by the star so that the star
                need not contract. What we have \emph{not} done yet is to derive the rate of heat loss from the star.
\section{Heat Transfer in Stars}
	In studying the ways in which heat moves outward through a star, we will first be ignoring convection, though it is a powerful mechanism when it is available to the star. To move heat through a star, there are electrons, ions, and photons available. Recall (or perhaps you don't) \textbf{Fick's Law}, which tells us that the heat flux can be determined via
	\begin{equation}
		\label{HTS.1} F=\mathrm{ergs\,cm^{-2}\,s^{-1}}=-\frac{1}{3}v\ell\frac{d}{dx}(E)
	\end{equation}
	where here $E$ is the energy density. We'll start first with
        the energy density of an ideal gas of electrons, but before we
        do that, let's derive \eqref{HTS.1}.
        \subsection{Heat Flux Derivation (not done in class)}
        Imagine a medium with a gradient in temperature across a surface
membrane. Let's say ``above'' the membrane, the temperature is $T_1$
and ``below'' the membrane, the temperature is $T_2$, with
$T_2>T_1$. Particles from region 2 then transport heat when they
travel from region 2 to region 1. Let's call $E$ the internal energy
per unit volume as defined before, and then let's see what happens at
the surface. Particles, on average, will be coming from a distance
$x+\ell$ above the membrane, where $\ell=(\sigma n)^{-1}$ is the mean
free path of the particles. Then the downward flow of energy is 
\begin{equation}
  \label{eq:3}
  F_{\mathrm{down}}\approx \frac{1}{6} vE(x+\ell)
\end{equation}
whereas particles from beneath move upward and carry heat from below
at
\begin{equation}
  \label{eq:1}
  F_{\mathrm{up}}\approx \frac{1}{6}vE(x-\ell)
\end{equation}
Think of the factor of $1/6$ as being the portion of the flux moving
through a particular face of a cube, in this case, the face pointing
up or down. So the net flux in the positive $\hat{x}$ direction is
\begin{equation}
  \label{eq:1a}
  F_x=-\frac{1}{6}vE(x+\ell)+\frac{1}{6}vE(x-\ell)
\end{equation}
Writing the energy densities as linear functions,
\begin{align}
  \label{eq:2}
  E(x+\ell) &= E(x)+\ell\frac{dE}{dx}\\
  \label{eq:2a}
  E(x-\ell) &= E(x)-\ell\frac{dE}{dx}
\end{align}
so
\begin{equation}
  \label{eq:4}
  F_x=-\frac{1}{3}v\ell\frac{dE}{dx}
\end{equation}
	\subsection{Heat Transport by Electrons} For electrons, the energy density is
	\begin{equation}
		\label{HTS.2} E=\frac{3}{2} kT n_e
	\end{equation}
	Then the energy density gradient is
	\begin{equation}
		\label{HTS.3} \frac{dE}{dx}=\frac{dE}{dT}\frac{dT}{dx}=\frac{3}{2}nk\frac{dT}{dx}
	\end{equation}
	Then from Fick's Law, we have
	\begin{equation}
		\label{HTS.4} F=-\frac{1}{3}v\ell\frac{3}{2}nk\frac{dT}{dx}=-\frac{1}{2}v\ell nk\frac{dT}{dx}
	\end{equation}
	Where the mean free path is
	\begin{equation}
		\label{HTS.5} \ell=\frac{1}{n\sigma}
	\end{equation}
	for the scattering cross section $\sigma$. Then \eqref{HTS.4} becomes
	\begin{equation}
		\label{HTS.6} F=-\left[\frac{1}{2} v\frac{k}{\sigma}\right]\bm{\nabla}T
	\end{equation}
	From the theory of Coulomb scattering, the cross section would be
	\begin{equation}
		\label{HTS.7} \sigma_{\mathrm{Coulomb}}\sim b^2\sim\frac{e^4}{(kT)^2}
	\end{equation}
	Which gives a flux of
	\begin{equation}
		\label{HTS.8} L=4\pi R^2F=4\pi R\frac{(kT)^{7/2}}{m_e^{1/2}e^4}
	\end{equation}
	For the sun, this would give
	\begin{equation}
		\label{HTS.9} L\approx 5\times 10^{31}\,\mathrm{erg\,s^{-1}}
	\end{equation}
	which is two orders of magnitude too small, so we conclude that the sun does not transmit its heat via conduction through electrons. Now we'll move on to photons
	\subsection{Radiative Diffusion}
	For photons, the main scatterer will be electrons, so the cross section in question of the mean free path is the Thomson cross section. Additionally, the energy density is now $E=aT^4$. Additionally the speed of photons is obviously the speed of light. Then the ratio of the fluxes due to photons and electrons is
	\begin{equation}
		\label{HTS.10} \frac{F_{\gamma}}{F_e}=\frac{c}{(kT/m_e)^{1/2}}\frac{e^4/(kT)^2}{\sigma_{\mathrm{Th}}}\frac{E_{\mathrm{rad}}}{E_{\mathrm{gas}}}
	\end{equation}
	Remember that the Thomson cross section is given by
	\begin{equation}
		\label{HTS.11} \sigma_{\mathrm{Th}}=\frac{8\pi}{3}\frac{e^4}{(m_ec^2)^2}
	\end{equation}
	Then \eqref{HTS.10} becomes
	\begin{equation}
		\label{HTS.12} \frac{F_\gamma}{F_e}=\left(\frac{m_ec^2}{kT}\right)^{1/2}\left(\frac{m_ec^2}{kT}\right)^2\frac{P_{\mathrm{rad}}}{P_{\mathrm{gas}}}
	\end{equation}
	Comparing the pressures gives
	\begin{equation}
		\label{HTS.13} \frac{P_{\mathrm{rad}}}{P_{\mathrm{gas}}}\approx 10^{-4}\left(\frac{M}{M_\odot}\right)^2
	\end{equation}
	Then plugging \eqref{HTS.13} into \eqref{HTS.12}, we see that heat transport by photons dominates heat transport by electrons whenever
	\begin{equation}
		\label{HTS.14} M>0.03\,M_\odot\,\left(\frac{T}{10^7\ \mathrm{K}}\right)^{5/4}
	\end{equation}
	So if conduction ever dominates, it is in very low mass stars (also white dwarfs in their late lives). For our cases, photons are always going to be the dominant transport mechanism. We still haven't found out what the actual luminosity will be in the case of radiative diffusion. We do so now.
	\begin{equation}
		\label{HTS.15} L=4\pi R^2 F=4\pi R^2\frac{1}{3}c\ell\frac{d}{dr}\left(aT^4\right)\approx R^2c\frac{1}{n_e\sigma_{\mathrm{Th}}}\frac{1}{R}aT^4\approx R^2\frac{c m_p}{\rho\sigma_{\mathrm{Th}}}\frac{1}{R}a\left(\frac{GMm_p}{Rk}\right)^4
	\end{equation}
	where we have noted that $kT=\frac{GMm_p}{R}$. Continuing on,
	\begin{equation}
		\label{HTS.16} L\approx \frac{cm_pa(GMm_p)^4}{M\sigma_{\mathrm{Th}}k^4}\propto M^3
	\end{equation}
	This relation is surprisingly accurate for stars with masses greater than a solar mass. Note that we have derived the stellar luminosity with \emph{no knowledge} of the source of energy. The luminosity is set by the modes of heat transport available to the star.\\
	
%%%%%	
	\n \textit{Monday, January 23, 2012}
	\subsubsection{The Eddington Limit and the Eddington Standard Model}
	Continuing with heat transfer via radiation with electron scattering being the primary source of opacity, the flux is given via Fick's law as
	\begin{equation}
		\label{HTS.17} F=\frac{1}{3} v\ell\frac{d}{dz}(aT^4)
	\end{equation}
	where now $v=c$ and $\ell=(n_e\sigma_{\mathrm{Th}})^{-1}$. Plugging these in to \eqref{HTS.17}, the flux becomes
	\begin{equation}
		\label{HTS.18} F=\frac{1}{3}c\frac{1}{n_e\sigma_{\mathrm{Th}}}\frac{d}{dz}(aT^4)=\frac{4}{3}\frac{acT^3}{\rho\kappa}\frac{dT}{dz}
	\end{equation}
	where we've defined the \textbf{opacity} via
	\begin{equation}
		\label{HTS.19} \kappa_{\mathrm{es}}\equiv\frac{\sigma_{\mathrm{Th}}}{m_p}
	\end{equation}
	The opacity is measured in units of area per unit mass, indicating it is the cross-sectional area per unit mass. For electron scattering, it is simply a constant, but when other processes are relevant, it may depend on temperature and density. With the flux available, we can write the luminosity as
	\begin{equation}
		\label{HTS.20} L(r)=F(r)4\pi r^2=-4\pi r^2\frac{4}{3}\frac{acT^3}{\rho\kappa}\frac{dT}{dr}=-4\pi r^2\frac{c}{\kappa\rho}\frac{d}{dr}P_{\mathrm{rad}}
	\end{equation}
	Noting the product $\rho\,dr$ showing up in the denominator of \eqref{HTS.20}, we are reminded of hydrostatic equilibrium:
	\begin{equation}
		\label{HTS.21} \frac{dP}{dr}=-\rho(r)g(r)\quad \Rightarrow \quad dP=-\rho(r)g(r)\,dr
	\end{equation}
	Putting in $g(r)$ explicitly,
	\begin{equation}
		\label{HTS.22} \frac{dP}{\rho(r)dr}=-\frac{Gm(r)}{r^2}
	\end{equation}
	Bringing the column depth back in ($y=\int \rho(r)\,dr$), this becomes
	\begin{equation}
		\label{HTS.23} \frac{dP}{dy}=\frac{Gm(r)}{r^2}
	\end{equation}
	where we've used the fact that $y=0$ at the surface and increases \emph{inwards} (hence the sign change). Assuming $g$ is a constant, this would give our old result
	\begin{equation}
		\label{HTS.24} \boxed{P=gy}
	\end{equation}
	Now returning to \eqref{HTS.20}:
	\begin{equation}
		\label{HTS.25} \frac{dP_{\mathrm{rad}}}{dy}=\frac{L(r)}{4\pi r^2}\frac{\kappa}{c}
	\end{equation}
	and the result we just derived
	\begin{equation}
		\label{HTS.26} \frac{Gm(r)}{r^2}
	\end{equation}
	Taking the ratio of \eqref{HTS.25} and \eqref{HTS.26} gives us
	\begin{equation}
		\label{HTS.27} \frac{dP}{dP_{\mathrm{rad}}}=\frac{4\pi Gm(r)c}{\kappa L(r)}
	\end{equation}
	Suppose $M=m(R)=\textrm{total mass of star}$ and $L=L(R)=\textrm{luminosity of star}$
	\begin{equation}
		\label{HTS.28} \frac{dP}{dP_{\mathrm{rad}}}=\left[\frac{4\pi GcM}{\kappa L}\right]\frac{m(r)}{M}\frac{L}{L(r)}
	\end{equation}
	We have now introduced the \textbf{Eddington Luminosity}, 
	\begin{equation}
		\label{HTS.29} L_{\mathrm{Edd}}=\frac{4\pi GcM}{\kappa}
	\end{equation}
	In general, the Eddington Luminosity is much larger than the luminosity, since it is the luminosity where the only pressure gradient that matters is the radiation pressure (where the star becomes unstable). Evaluating the Eddington Luminosity in solar units and with electron scattering,
	\begin{equation}
		\label{HTS.30} L_{\mathrm{Edd}}=3.1\times 10^4\ L_{\odot}\left(\frac{M}{M_\odot}\right)
	\end{equation}
	Also recall how luminosity scales with mass:
	\begin{equation}
		\label{HTS.31} L=L_\odot\left(\frac{M}{M_\odot}\right)
	\end{equation}
	So we see that for low-mass stars, their luminosities are indeed much smaller than the Eddington Luminosity. Now let's investigate the ratio of the luminosity to the Eddington Luminosity:
	\begin{equation}
		\label{HTS.32} \frac{L}{L_{\mathrm{Edd}}}\approx 3\times 10^{-5}\left(\frac{M}{M_\odot}\right)^2
	\end{equation}
	So until $M\geq 100\,M_\odot$, it will be the case that $L\ll L_{\mathrm{Edd}}$ and thus $P_{\mathrm{rad}}\ll P$. Now returning to \eqref{HTS.28}, we define a dimensionless quantity, $\eta(r)$ by
	\begin{equation}
		\label{HTS.33} \eta(r)\equiv \frac{L(r)}{L}{M}{m(r)}
	\end{equation}
	And \eqref{HTS.28} looks like
	\begin{equation}
		\label{HTS.34} \frac{dP_{\mathrm{rad}}}{dP}=\frac{L}{L_{\mathrm{Edd}}}\eta(r)
	\end{equation}
	Aside: this model of stars is the called the ``Eddington Standard Model'' and was used to describe stars before their source of energy was known. 
	\subsubsection{Polytropic Relations}
	Now integrating \eqref{HTS.34}, we have
	\begin{equation}
		\label{HTS.35} \int_R^rdP_{\mathrm{rad}}=\frac{L}{L_{\mathrm{Edd}}}\int_R^r\eta(r)\,dP
	\end{equation}
	We can integrate the left side, leaving us with
	\begin{equation}
		\label{HTS.36} P_{\mathrm{rad}}(r)=\frac{L}{L_{\mathrm{Edd}}}\int_R^r\eta(r)\,dP
	\end{equation}
	Mathematically speaking, we are now stuck because we do not have any knowledge of $\eta(r)$. The formal approach of solving this problem is to define spatial averages of $\eta(r)$ for different choices of mass and luminosity profiles. We will, however, just let $\eta(r)\approx 1$, which gives us
	\begin{equation}
		\label{HTS.37} P_{\mathrm{rad}}=\frac{L}{L_{\mathrm{Edd}}}P_{\mathrm{gas}}(r)
	\end{equation}
	And now substituting in our equations of state for the radiation pressure and assuming that the radiation pressure is negligible compared to the gas pressure:
	\begin{equation}
		\label{HTS.38} \frac{1}{3}aT^4=\frac{L}{L_{\mathrm{Edd}}}\frac{\rho kT}{\mu m_p}
	\end{equation}
	Solving this for $T^3$, we have
	\begin{equation}
		\label{HTS.39} \boxed{T(r)^3=\frac{L}{L_{\mathrm{Edd}}}\frac{3k}{a\mu m_p}\rho(r)}
	\end{equation}
	So we have gone from a situation where we had only typical or central values to an actual equation that let's us find $T(r)$, $\rho(r)$, and $P(r)$. It appears then, that in our low radiation pressure limit, $\rho\propto T^3$. Additionally, for the ideal gas law, we see that $P\propto \rho^{4/3}$. For some given $L/L_{\mathrm{Edd}}$, we have a relation between pressure and density via
	\begin{align}
		\label{HTS.40} \frac{dP}{dr}&=-\rho(r)\frac{Gm(r)}{r^2}\\
		\label{HTS.41} dm(r) &= \rho 4\pi r^2\,dr
	\end{align}
	Let us briefly consider an adiabatic change in an ideal gas. We recall from Freshman physics that $PV^\gamma=\mathrm{const}$. For a monatomic, ideal gas, $\gamma=5/3$, so we have $P\propto \rho^{5/3}$ and $\rho T\propto P$, so $T\propto \rho^{2/3}$. Then we'll call the ``entropy'' $T/\rho^{2/3}$. For our star, we have $T^3\propto \rho\Rightarrow T\propto \rho^{1/3}$. In our case, the ``entropy'' is then $a/\rho^{1/3}$. Thus the entropy is highest in the outer layers and smallest in the center.
	\subsubsection{Heat Transfer in the Outer Atmosphere}
	Near the surface of the star, photons naturally leave (which is why we see them!) We want to know what the place or condition is like. Near the surface, the density must decrease exponentially since $T\ll T_c$ and thus $g$ is a constant. Our previous arguments for the scale height and the plane-parallel isothermal atmosphere is largely applicable here, so the length scale of relevance is the scale height:
	\begin{equation}
		\label{HTS.42} h=\frac{kT}{\mu m_p g}
	\end{equation}
	It then makes sense that the condition for photons to escape would be for the mean free path to be comparable to the scale height:
	\begin{equation}
		\label{HTS.43} \ell\sim h\quad\Rightarrow\quad \frac{m_p}{\rho\sigma_{\mathrm{Th}}}\approx \frac{kT}{mg}
	\end{equation}
	or
	\begin{equation}
		\label{HTS.44} g\frac{m_p}{\sigma_{\mathrm{Th}}}\sim\frac{\rho kT}{m_p}\sim P_{\mathrm{gas}}
	\end{equation}
	Again, this can be rewritten as
	\begin{equation}
		\label{HTS.45} g\kappa^{-1}\sim P_{\mathrm{gas}}\quad\mathrm{where}\quad \kappa=\frac{\sigma_{\mathrm{Th}}}{m_p}
	\end{equation}
	Then we can say the condition where photons can escape is
	\begin{equation}
		\label{HTS.46} P_{\mathrm{gas}}\leq \frac{g}{\kappa}
	\end{equation}
	This argument assumed a constant opacity. In general, we must do a line integral through the depth of the outer atmosphere. Imagine we ask the probably for a photon to reach some depth in the star. We define the \textbf{optical depth} to be
	\begin{equation}
		\label{HTS.47} d\tau = \frac{dr}{\ell}=dr\ \kappa(r)\rho(r)
	\end{equation}
	Then the probability would be
	\begin{equation}
		\label{HTS.48} P(\tau)\propto e^{-\tau}
	\end{equation}
	Then the optical depth at a certain radius would be given by
	\begin{equation}
		\label{HTS.49} \tau = \int_R^r\kappa(r)\rho(r)\,dr
	\end{equation}
	or, invoking hydrostatic equilibrium,
	\begin{equation}
		\label{HTS.50} \tau=\int \kappa(r)\frac{dP(r)}{g}=\frac{1}{g}\int \kappa(r)\,dP(r)
	\end{equation}
	Then the condition for photons for leave becomes when the optical depth is unity, or
	\begin{equation}
		\label{HTS.51} \tau\sim1=\frac{1}{g}\kappa P(r)\quad\Rightarrow\quad P(\tau=1)=\frac{g}{\kappa}
	\end{equation}
	Additionally, we can redefine the flux equation in terms of optical depth into a cute form:
	\begin{equation}
		\label{HTS.52} F=\frac{c}{3}\frac{d}{d\tau}\left(aT^4\right)
	\end{equation}
	\subsubsection{The Effective Temperature}
	At the surface, the flux is given by
	\begin{equation}
		\label{HTS.53} F=\frac{L}{4\pi R^2}\equiv \sigma_{\mathrm{SB}}T_{\mathrm{eff}}^4
	\end{equation}
	Assuming that the flux is constant at or near the surface, we may use the fact that $a=4\sigma_{\mathrm{SB}}/c$ and \eqref{HTS.52} to get
	\begin{equation}
		\label{HTS.54} \frac{d}{d\tau}T^4=\frac{3}{4}T_{\mathrm{eff}}^4
	\end{equation}
	Recall that $\tau$ is dimensionless, so there is no problem here. For $\tau\gg 1$, we have
	\begin{equation}
		\label{HTS.55} T^4=\frac{3}{4}T_{\mathrm{eff}}^4\tau
	\end{equation}
	This implies that as $\tau$ increases (going deeper and deeper
        into the star), the radiation field becomes more and more
        isotropic. Naively, we might assume that the flux would be
        given by the energy density multiplied by the speed of light,
        but our result in \eqref{HTS.55} essentially tells us that
        $F=acT^4/\tau$.\\

        \n \textit{Wednesday, January 25, 2012}
\subsection{Convection}
\label{sec:conv}
Another important form of heat transport in stars is that of
convection, which is where an instability causes the bulk movement of
material (rather than the diffusion of photons or electrons) to
transport heat throughout the star. Convection occurs only when the
temperature gradient is very steep. Otherwise, radiative diffusion
dominates heat transport.\\

\n The origin of the instability is that a fluid element that rises
adiabatically (faster than the heat transfer time scale) is
\textbf{lighter} than the surrounding fluid and so it
``runs away''. This is a linear instability, in that a displacement
exponentially grows in time. \\

\n Suppose we ahve a fluid element in hydrostatic balance. We imagine
displacing this fluid element from $r$ to $r+dr$. So we assume that
the displaced element responds adiabatically. Thermodynamcis tells us
that
\begin{equation}
  \label{eq:9}
  T\,dS=dE+P\,dV=0
\end{equation}
for an adiabatic process (here these quantities are related to the
inside of the fluid element). This is equivalent to the requirement that
the timescale of response is much less than that for heat to enter or
leave the fluid element. We presume that the timescale is longer than
the time for pressure equilibrium as well:
\begin{equation}
  \label{eq:10}
  t=\frac{h}{c_s}=\textrm{time for the pressure to
    equilibrate}\quad\Rightarrow\quad v\ll c_s
\end{equation}
where $h$ is agin the scale height. This requirement ensures that the
fluid element remains in pressure equilibrium with its surroundings at
all times Suppose the element starts in
location 1 and travels to location 2. This adiabatic process requires
that $PV^\gamma$ be a constant within the bubble, which means that
\begin{equation}
  \label{eq:11}
  \frac{P_1}{\rho_1^\gamma}=\frac{P_2}{\rho_2^\gamma}
\end{equation}
where the subscripts refer to where the bubble is located (and the
quantities are measured within the bubble, not the surrounding
material). Rearranging
\eqref{eq:11} gives us the new density
\begin{equation}
  \label{eq:12}
  \rho_{2,b}^\gamma=\rho_1^\gamma\frac{P_2}{P_1}
\end{equation}
Clearly the bubble is now less dense than it was at its starting
position, but how dense is it compared to the surrounding material?
For \emph{stability}, we require $\rho_{2,b}>\rho_{2,*}$, where the * indicates the
density of the star's ``background'' conditions. For the star, we
require
\begin{equation}
  \label{eq:13}
  \rho_1\left(\frac{P_2}{P_1}\right)^{1/\gamma}>\rho_{2,*}
\end{equation}
Note that $\rho_1$, $P_1$, and $P_2$ are the same for both the fluid
element and the background. For an infinitesimal change, the pressures
are related via
\begin{equation}
  \label{eq:14}
  P_2=P_1+\Delta r\frac{dP}{dr}
\end{equation}
and likewise the densities, 
\begin{equation}
  \label{eq:15}
  \rho_2=\rho_1+\frac{d\rho}{dr}\Delta r
\end{equation}
(where all quantities are for the star, not the fluid element, as they
shall be from now on unless otherwise noted). Then \eqref{eq:13} requires
\begin{equation}
  \label{eq:16}
  \frac{1}{\gamma}\frac{d\ln P}{d\ln r}>\frac{d\ln\rho}{dr}
\end{equation}
Now, we recall that
\begin{equation}
  \label{eq:17}
  P=\frac{\rho kT}{\mu m_p} \quad\Rightarrow \quad d\ln\rho=d\ln
  P-d\ln T
\end{equation}
(assuming uniform composition) and thus \eqref{eq:16} can be written as
\begin{equation}
  \label{eq:18}
  \frac{1}{\gamma}\frac{d\ln P}{dr}>\frac{d\ln P}{dr}-\frac{d\ln T}{dr}
\end{equation}
For an ideal, monatomic gas, this becomes
\begin{equation}
  \label{eq:19}
  \frac{d\ln T}{d\ln P}<1-\frac{1}{\gamma}=\frac{2}{5}
\end{equation}
Recall that earlier we found that for radiative heat transport in the
Eddington standard model,
$\rho\propto T^3$, and thus $P\propto \rho T\propto T^4$, so
$T\propto P^{1/4}$. Then computing the logarithmic derivative in
\eqref{eq:19}, 
\begin{equation}
  \label{eq:20}
  \frac{d\ln T}{d\ln P}=\frac{1}{4}<\frac{2}{5}
\end{equation}
Note that we could also include a gradient in the composition if we
wanted. \eqref{eq:20} would become, more genrally,
\begin{equation}
  \label{eq:135}
  \boxed{\left(\frac{1}{\gamma}-1\right)\frac{d\ln
    P}{dr}-\frac{d\ln\mu}{dr}>-\frac{d\ln T}{dr}}
\end{equation}
This actually weakens the requirement on the temperature
gradient. \S6.3 of Keppenhahn and Weigert discuss the strange
consequences of being in the ``twilight zone'' between the conditions
set up by \eqref{eq:135} and \eqref{eq:19}. It is possible, in this
regime, to have a fluid element oscillate (as in a stable case), but
have the oscillation grow in amplitude (not exactly what we would
think of as stable).\\

\n Barring these strange situations, stars that follow \eqref{eq:135}
and certainly those following \eqref{eq:19} are stable and do not transport heat by
convection. A model that is stable to convection has the entropy
increasing with radius. Note that when we say ``entropy'', we mean the
\textbf{specific entropy}, or the entropy per unit mass.\\

\n To put things in context, we recall the Eddington standard model,
where we found that $\rho\propto T^3$, so $P\propto T^4$. Then we have
$d\ln T/d\ln P=1/4<2/5$. So then the Eddington standard model star is
stable to convection and thus transports heat primarily by radiative
diffusion.
\subsubsection{The Unstable Case}
\label{sec:unstable-case}

Now we wish to understand what happens when the background model is
unstable. The density in the bubble is
\begin{equation}
  \label{eq:21}
  \rho_{2,b}=\rho_1\left(\frac{P_2}{P_1}\right)^{1/\gamma}
\end{equation}
and for the star,
\begin{equation}
  \label{eq:22}
  \rho_{2,*}=\rho_1+\left.\Delta r\frac{d\rho}{dr}\right|_*
\end{equation}
The density contrast is then
\begin{equation}
  \label{eq:23}
  \Delta \rho=\rho_{2,*}-\rho_{2,b}=\Delta r\left[\left.\frac{d\rho}{dr}\right|_*-\frac{1}{\gamma}\frac{\rho}{P}\left.\frac{dP}{dr}\right|_*\right]
\end{equation}
or
\begin{equation}
  \label{eq:24}
  \Delta\rho=\rho\Delta r\left[\frac{d\ln\rho}{dr}+\frac{\rho g}{P\gamma}\right]
\end{equation}
where we have invoked hydrostatic equilibrium. Stability requires
$\Delta\rho<0$, or
\begin{equation}
  \label{eq:25}
  \boxed{\frac{d\ln\rho}{dr}<-\frac{\rho g}{\gamma P}\quad\textrm{(Stability)}}
\end{equation}
The is the same result we obtained previously, just phrased a bit
differently. In the unstable case, though, we require $\Delta\rho>0$. the
acceleration, $a$, or the displaced element will be given by 
\begin{equation}
  \label{eq:26}
  a=\frac{\Delta\rho}{\rho}g=g\Delta
  r\left[\frac{d\ln\rho}{dr}+\frac{\rho g}{P\gamma}\right]
\end{equation}
This is the equation of motion for the fluid element. Written in
terms of the displacement coordinate $x$, \eqref{eq:26} is
\begin{equation}
  \label{eq:27}
  \ddot{x}=gx\left(\frac{d\ln\rho}{dr}+\frac{\rho g}{P\gamma}\right)
\end{equation}
Strictly speaking this equation holds for unstable or stable
cases. Clearly this is a harmonic oscillator, so the physics should be
familiar. In any case, we have
\begin{equation}
  \label{eq:28}
  -\omega^2=g\left(\frac{d\ln\rho}{dr}+\frac{\rho g}{P \gamma}\right)
\end{equation}
If $-\omega^2<0$, then the solution is stable, and the element
oscillates at the Brunt V\"{a}is\"{a}l\"{a} frequency is 
\begin{equation}
  \label{eq:29}
  N^2=-g\left[\frac{d\ln\rho}{dr}+\frac{\rho g}{\gamma P}\right]
\end{equation}
This is the local frequency of oscillation of a fluid element in a
convectively stable atmosphere. In the Earth's atmosphere, this
frequency is around five or ten minutes.\\

\n Now if this coefficient is negative, the solution is unstable
(i.e., groes exponentially). Note that $N^2\sim g/h$, so 
\begin{equation}
  \label{eq:30}
  N^2\sim\frac{gm_p
    g}{kT}\sim g^2\frac{1}{v_{\mathrm{th}}^2}\quad\Rightarrow N\sim \frac{g}{v_{\mathrm{th}}}\sim\frac{g}{c_s}
\end{equation}
So our displacement solution looks like
\begin{equation}
  \label{eq:31}
  x=x_0e^{t/\tau}
\end{equation}
where $1/\tau^2=-N^2$. So the speed is 
\begin{equation}
  \label{eq:32}
  \dot{x}=\frac{x_0}{\tau}e^{t/\tau}=\ell/\tau=\textrm{speed after
    moving a distance $\ell$}
\end{equation}
Generally, the speed is given by
\begin{equation}
  \label{eq:33}
  v=\ell\sqrt{-N^2}
\end{equation}
Suppose that the stellar interior is strongly unstable, so that
$N^2\sim-g/h$. Then the velocity after traveling some length $\ell$ is 
\begin{equation}
  \label{eq:34}
  v=\ell\left(\frac{g}{h}\right)^{1/2}=\ell\left(\frac{gm_pg}{kT}\right)^{1/2}=\ell\frac{g}{v_{\mathrm{th}}}\approx \frac{g}{v_{\mathrm{th}}}h=\frac{g}{v_{\mathrm{th}}}\frac{v_{\mathrm{th}}^2}{g}=v_{\mathrm{th}}
\end{equation}
So if $N^2\sim -g/h$, i.e., the star is strongly unstable, then a runaway
fluid element will reach the sound speed after traversing one scale
height! Recall, though that the change in density is
\begin{equation}
  \label{eq:35}
 \left. \frac{\Delta\rho}{\rho}\right|_{\mathrm{wt\ \ell}}\approx
 \ell\left(\frac{d\ln \rho}{dr}+\frac{\rho g}{P\gamma}\right)
\end{equation}
and thus the velocity is
\begin{equation}
  \label{eq:36}
  v=\ell\sqrt{\left.\frac{g}{\ell}\left(\frac{\Delta
        \rho}{\rho}\right)\right|_{\ell}}=(g\ell)^{1/2}\left(\left.\frac{\Delta \rho}{\rho}\right|_{\ell}\right)^{1/2}
\end{equation}
\textit{Monday, January 30, 2012}\\

\subsubsection{Convective Efficiency}
\label{sec:conv-effic}

  We left off last time with a rough relation between the velocity of
  an upwardly rising fluid element and the density contrast between
  the fluid element and the background material at some height $\ell$
  above the element's original location. In the unstable case, we
  found that the element was less dense than the surrounding fluid,
  by examining the quantity $\Delta\rho/\rho|_{\ell}$. The velocity at
  which the element traveled was given by
  $v=(g\ell)^{1/2}(\Delta\rho/\rho|_\ell)^{1/2}$.\\

  \n Inserting the scale height in for $\ell$,
  \begin{equation}
    \label{eq:37}
    \ell=h=\frac{kT}{mg}=\frac{c_s^2}{g}
  \end{equation}
  Then the velocity would be
  \begin{equation}
    \label{eq:38}
    v=c_s\left[\left.\frac{\Delta\rho}{\rho}\right|_h\right]^{1/2}
  \end{equation}
  Now we consider the flux due to convection. The flux is just the
  velocity times the energy density, and so we can write
  \begin{equation}
    \label{eq:39}
    F=v\rho v^2=\rho v^3
  \end{equation}
  Assuming $\Delta\rho/\rho\ll 1$, we may
  assume that the element is moving subsonically.\\

  \n The business of heat flux via convection is covered in
  \textbf{Mixing Length Theory}. In this ``theory'', we imagine a
  convective element traveling up a \textbf{mixing height}, $\ell$ and
  then allowing the fluid element to equilibrate with the
  surroundings. Since the preferred lengthscale in a star is the scale
  height, we typically choose $\ell_{\mathrm{MT}}=\alpha h$ for some
  dimensionless constant $\alpha$. Applying this to \eqref{eq:39},
  \begin{equation}
    \label{eq:40}
    F_{\mathrm{conv}}=\rho\frac{kT}{m_p}c_s\left(\left.\frac{\Delta\rho}{\rho}\right|_{\ell=h}\right)^{3/2}=Pc_s\left(\left.\frac{\Delta\rho}{\rho}\right|_{\ell=h}\right)^{3/2}
  \end{equation}
  Now recall our result for the radiative flux:
  \begin{equation}
    \label{eq:41}
    F_{\mathrm{rad}}=\frac{1}{3}\frac{c}{\kappa\rho}\frac{d}{dr}\left(aT^4\right)\approx \frac{cP_{\mathrm{rad}}}{\tau}
  \end{equation}
  where in the second equality, we're let $d/dr\to 1/h$. Now we find
  the ratio in these two fluxes:
  \begin{equation}
    \label{eq:42}
    \frac{F_{\mathrm{conv}}}{F_{\mathrm{rad}}}\sim \frac{\tau Pc_s\left(\left.\Delta\rho/\rho\right|_h\right)^{3/2}}{cP_{\mathrm{rad}}}\sim\frac{c_s}{c}\frac{P}{P_{\mathrm{rad}}}\tau\left(\left.\frac{\Delta\rho}{\rho}\right|_h\right)^{3/2}
  \end{equation}
  For typical, low-mass stars, we assume that $P/P_{\mathrm{rad}}\sim 10^4\gg
  1$, let's find when both fluxes give equal contributions (also
  plugging in some solar values):
  \begin{equation}
    \label{eq:43}
    \left(\frac{kT}{m_pc^2}\right)^{1/2}10^4\tau\left(\frac{\Delta\rho}{\rho}\right)^{3/2}\sim 1
  \end{equation}
  The requirement becomes
  \begin{equation}
    \label{eq:44}
    \frac{\Delta\rho}{\rho}\approx 2\left(\frac{10^4}{T}\right)^{1/3}\frac{1}{\tau^{2/3}}
  \end{equation}
  Where $\tau$ and $T$ get smaller, the simple theory implies
  convection at near the sound speed, $c_s$. In truth, you can't do
  this problem correctly in one dimension, so this result isn't all
  that good. What does this imply about the surface, though? Stellar
  convection at or near the surface becomes sonic, which we say is
  \textbf{inefficient} \emph{and} $\Delta\rho/\rho\to 1$.\\

  \begin{quote}
    What I'm trying to do here is to write a density model in 1-D that
    doesn't make me blush.\hfill \emph{--Lars Bildsten}
  \end{quote}
  To find a situation where this \emph{does} work is deep in the
  interior of the star, where $\tau \ggg 1$ (yes, three greater than
  signs). The relevant lengthscale here is no longer the scale height,
  which begins to approach the radial coordinate:
  \begin{equation}
    \label{eq:45}
    h=\frac{P}{\rho g}\sim\frac{GM^2}{R^4M}\frac{R^3R^2}{GM}\sim R
  \end{equation}
  so we'll just use the radius of the star as our characteristic
  length scale. The ratio of fluxes is now 
  \begin{equation}
    \label{eq:46}
    \frac{F_{\mathrm{conv}}}{F_{\mathrm{rad}}}\sim \frac{c_s}{R}\left[\frac{P}{P_{\mathrm{rad}}}\tau\frac{R}{c}\right]\left(\left.\frac{\Delta\rho}{\rho}\right|_{R}\right)^{3/2}
  \end{equation}
  Note that the time it takes for a photon to diffuse from a star is
  $t_{\mathrm{rw}}=\tau R/c$. Thus, that is the time it takes to
  evacuate the radiation field of energy, so multiplying by the ratio
  in pressures is actually the Kelvin-Helmholtz time. Using this fact,
  \eqref{eq:46} can be written as
  \begin{equation}
    \label{eq:47}
    \frac{F_{\mathrm{conv}}}{F_{\mathrm{rad}}}=\frac{t_{\mathrm{KH}}}{t_{\mathrm{dyn}}}\left(\left.\frac{\Delta\rho}{\rho}\right|_{R}\right)^{3/2}
  \end{equation}
  Deep in the star, $t_{\mathrm{KH}}\ggg t_{\mathrm{dyn}}$, so
  convection is efficient when
  \begin{equation}
    \label{eq:48}
    \left.\frac{\Delta\rho}{\rho}\right|_R\sim\left(\frac{t_{\mathrm{dyn}}}{t_{\mathrm{KH}}}\right)\lll 1
  \end{equation}
  When convection is efficient, then, the stellar model nearly follows
  the adiabatic relation! Deep interior convection, in this limit,
  only implies that we need to know the adiabatic relation very
  well. The punchline here is that in the deep interior of a star, if
  the temperature gradient is even slightly super-adiabatic,
  convection becomes an incredibly powerful form of heat transport.\\

  \n Note that the ratio of times scales like
  \begin{equation}
    \label{eq:49}
    \frac{t_{\mathrm{dyn}}}{t_{\mathrm{KH}}}\approx 3\times
    10^{-9}\left(\frac{M}{10\ M_\odot}\right)^{3.5}
  \end{equation}
  Noting that $L\propto M^{3.5}$ and $R=R_\odot(M/M_\odot)$ on the
  main sequence, this gives us
  \begin{equation}
    \label{eq:50}
    \frac{v}{c_s}\approx 10^{-3}\left(\frac{M}{10\ M_\odot}\right)^{7/6}
  \end{equation}
  This relatively slow speed reinforces our assumption that convective
  bubbles move slow enough to maintain pressure equilibrium with their
  surroundings. Now remember the adiabatic condition for convection
  derived earlier:
  \begin{equation}
    \label{eq:51}
    \left.\frac{d\ln T}{d\ln P}\right|_*=\left.\frac{d\ln T}{d\ln P}\right|_{\mathrm{adiabatic}}=\frac{2}{5}
  \end{equation}
  So in the convective zone, we must have $T\propto P^{2/5}\propto
  (\rho T)^{2/5}$, or more directly, $T\propto \rho^{2/3}$. So far, we
  have motivated \emph{two} distinct \textbf{polytropic relations}
  (models where $P\propto \rho^{(n+1)/n}$ for some $n$):
  \begin{enumerate}
  \item[1.] \textbf{Fully Convective and Efficient}: $P\propto \rho^{5/3}$,
    where the prefactor is the specific entropy of the star, which is
    spatially constant. Thus if we have the mass of a star and the
    entropy, we can calculate everything we want to know about this
    star (in this simple model).
  \item[2.] \textbf{Constant} $\bm{L/M}$ star with constant (electron scattering)
    opacity, where $P\propto \rho^{4/3}$.
  \end{enumerate}

  \subsection{Fully Convective Star}
  \label{sec:fully-conv-star}

  Imagine a star of ideal gas \emph{and} from the core to the
  photosphere is fully convective \emph{and} efficient (this will
  likely be a bad approximation near the surface). Somewhere near the
  surface, photons must be made to allow the energy to escape the
  star. Near the photosphere, the pressure is
  \begin{equation}
    \label{eq:52}
    P_{\mathrm{ph}}\approx \frac{g}{\kappa_{\mathrm{ph}}}
  \end{equation}
  If we were to plot $\ln T$ against $\ln P$ (as we did in lecture),
  for this star, it would have a slope of $2/5$. Getting out towards
  the surface, though, the curve would have to be a bit shallower. We
  hope this only happens out towards one scale height in from the
  surface. Incorporating this physics will change $R$, but only on the
  order of $\left.h\right|_{\mathrm{location}}$. Noting that
  \begin{equation}
    \label{eq:53}
    \frac{h}{R}\sim \frac{T}{T_c}
  \end{equation}
  we see this is an extremely small error, since $T/T_c$ is very small
  out towards the surface. We assume $P\propto \rho^{5/3}$ (an $n=3/2$
  polytrope). In this model, the central pressure is
  \begin{equation}
    \label{eq:54}
    P_c=0.77\frac{GM^2}{R^4}
  \end{equation}
  The central temperature is
  \begin{equation}
    \label{eq:55}
    T_c=0.54\frac{GM\mu m_p}{kR}
  \end{equation}
  Requiring the entropy be the same everywhere requires
  \begin{equation}
    \label{eq:56}
    \frac{T_c}{P_c^{2/5}}=\frac{T_{\mathrm{ph}}}{P_{\mathrm{ph}}^{2/5}}=\frac{T_{\mathrm{eff}}}{P_{\mathrm{ph}}^{2/5}}
  \end{equation}
  Since we've assumed the perfect adiabat (admittedly a lie\ldots\
  also, $\mu$ is changing due to varying ionization conditions in the
  outer layers of the star). The surface thermal energy is
  \begin{equation}
    \label{eq:57}
    kT_{\mathrm{eff}}=0.6\left(\frac{GM\mu m_p}{R}\right)\left(\frac{R^2}{M\kappa_{\mathrm{ph}}}\right)^{2/5}
  \end{equation}
  Putting in $\kappa_{\mathrm{ph}}=\kappa_{\mathrm{es}}$ gives us 
  \begin{equation}
    \label{eq:58}
    T_{\mathrm{eff}}=200\ \mathrm{K}\left(\frac{M}{M_\odot}\right)^{3/5}\left(\frac{R_\odot}{R}\right)^{1/5}
  \end{equation}
  So we have a big problem with assuming electron scattering (or
  convection) since this is obviously too low for a star like the
  sun.\\
%%%%%%%%%%%%%%%%%%%%%%%%%%%%%%%
% Wednesday, February 1, 2012 %
%%%%%%%%%%%%%%%%%%%%%%%%%%%%%%%
  \n\textit{Wednesday, February 1, 2012}\\
  \subsection{The Hayashi Track}
  \label{sec:fully-conv-stars}
  \n It turns out that in this case, the outer body sets the
  temperature. Electron scattering is \emph{not} the primary source of
  opacity. Typically $\mathrm{H^-}$ opacity is much more
  important. This is where a very loosely bound ion (ionization energy
  oround 0.75\ eV) ``ionized'' by a photon (really, it's un-ionized)
  in the reaction
  \begin{equation}
    \label{eq:136}
    \gamma+\mathrm{H^-}\to e^{-}+\mathrm{H}
  \end{equation}
  Hansen, Kawaler, and Trimble give the value of this opacity as
  \begin{equation}
    \label{eq:137}
    \kappa_{\mathrm{H^-}}=2.5\times 10^{-31}\rho^{1/2} T^9\
    \mathrm{cm^2\ g^{-1}}
  \end{equation}
  Now with the pressure given by
  \begin{equation}
    \label{eq:138}
    P=\frac{\rho kT}{\mu m_p}
  \end{equation}
  and the density at the photosphere given by
  \begin{equation}
    \label{eq:139}
    \rho=\frac{g}{\kappa}\frac{\mu m_p}{kT}
  \end{equation}
  Combining \eqref{eq:138} and \eqref{eq:139} gives the density as
  \begin{equation}
    \label{eq:140}
    \rho=\frac{g\mu m_p}{kT \kappa_0 \rho^{1/2}T^9}
  \end{equation}
  Or we could have 
  \begin{equation}
    \label{eq:141}
    \rho^{3/2}=\frac{g\mu m_p}{kT^{10}\kappa_0}\quad\Rightarrow\quad
    \rho=\left(\frac{g\mu m_p}{kT^{10}\kappa_0}\right)^{2/3}
  \end{equation}
  Comparing these two values for the densities gives an opacity of
  \begin{equation}
    \label{eq:142}
    \kappa_{\mathrm{H^-}}=\kappa_0 T^9\left(\frac{g\mu m_p}{k T^{10}\kappa_0}\right)^{1/3}
  \end{equation}
  or more succinctly,
  \begin{equation}
    \label{eq:143}
    \boxed
    \kappa_{\mathrm{H^-}}=\kappa_0^{2/3}T^{17/3}\left(\frac{g\mu m_p}{k}\right)^{1/3}
  \end{equation}
  Using this opacity in \eqref{eq:57}, we obtain (omitting some
  algebra), the more reasonable result of
  \begin{equation}
    \label{eq:144}
    T_{\mathrm{eff}}\approx 2500\ \mathrm{K}\left(\frac{M}{M_{\odot}}\right)^{1/7}\left(\frac{R}{R_\odot}\right)^{1/49}
  \end{equation}
  So the luminosity oges as
  \begin{equation}
    \label{eq:145}
    \frac{L}{L_\odot}=0.034\left(\frac{M}{M_\odot}\right)^{4/7}\left(\frac{R}{R_\odot}\right)^{102/49}
  \end{equation}
  Writing the effective temperature as a function of the mass and the
  luminosity gives us
  \begin{equation}
    \label{eq:146}
    T_{\mathrm{eff}}\approx 2600\ \mathrm{K}\left(\frac{L}{L_\odot}\right)^{1/102}\left(\frac{M}{M_\odot}\right)^{7/51}
  \end{equation}
  Note that this is nearly independent of the luminosity, so it is
  nearly a straight vertical line on the HR diagram. As a molecular
  cloud contracts, heat transport is dominated by convection and the
  opacity in the photosphere is dominated by $\mathrm{H}^{-}$
  opacity. This explains how we see stars descending a vertical line
  in the HR diagram called the \textbf{Hayashi Track} that corresponds
  to \eqref{eq:146}. Note, however, that there is a lower limit on the
  effective temperature below which no hydrostatic radiating solutions
  exist because it is impossible to have the same entropy in the
  photosphere as in the core (recall that that was a requirement for
  convection to take place).\\

  \n The contraction of a protostar and its corresponding trip down
  the Hayashi Track is rather quick at the high-luminosity end but
  changes as the star contracts. So the initial evolution is down the
  Hayashi Track until a radiative solution exists, after which point
  the star evolves along a nearly constant constant luminosity track
  (the less-famous Henyey Track).

\section{Star Formation and the Jean's Mass}
\label{sec:star-formation-jeans}
After discussing a star's trip down the Hayashi track, we take a
moment to investigate a bit more about the physics of star formation,
which is still a field with many unanswered questions.
\subsection{The Jean's Mass}

  In the interstellar medium, there are large masses of cold dust and
  gas that can give rise to isolated regions of stellar formation. Often
  these manifestations are called \textbf{cold cores}. Suppose one of
  these cold cores is in the ISM with some density $\rho$ and
  temperature $T$ (likely around 10-20 K). The gravitational energy is
  something like
  \begin{equation}
    \label{eq:59}
    E_{\mathrm{GR}}\approx -\frac{GM^2}{R}
  \end{equation}
  And the thermal content is something like
  \begin{equation}
    \label{eq:60}
    E_{\mathrm{th}}=\frac{3}{2}\frac{M}{m_p}kT
  \end{equation}
  In order for this cloud to collapse, we must have the total energy
  being less than zero. Thus, we must have
  \begin{equation}
    \label{eq:61}
    \frac{GM^2}{R}>\frac{3}{2}\frac{M}{m_p}kT
  \end{equation}
  If we assume a constant density profile, this tells us that the mass
  of the object must be
  \begin{equation}
    \label{eq:62}
    M>500\ M_\odot\left(\frac{T}{10\
        \mathrm{K}}\right)^{3/2}\left(\frac{1\ \mathrm{cm}^{-3}}{n}\right)^{1/2}
  \end{equation}
  This mass is called the \textbf{Jean's Mass}. If a cloud is more
  massive than this, it is possible to collapse. Now, it may stay as
  one large blob, or it may fragment\ldots we're not sure yet. The
  question is then, ``How does the Jean's mass change in a collapsing
  cloud?'' We know that the Jeans mass scales as
  \begin{equation}
    \label{eq:63}
    M_\mathrm{J}\propto T^{3/2}\rho^{-1/2}
  \end{equation}
  As the star collapses the density must increase. \emph{If} the
  material maintains its entropy, then we must have $T\propto
  \rho^{2/3}$. Then the Jean's mass scales as
  \begin{equation}
    \label{eq:64}
    M_{\mathrm{J}}\propto T^{3/2}\rho^{-1/2}\propto\rho^{1/2}
  \end{equation}
  So if the collapse is adiabatic, the collapse does not lead to
  fragmentation. The cloud will reach some critical density, at which
  point the Jean's mass is higher than the mass of the cloud, halting
  collapse.\\

  \n Cooling of the gas (i.e., the temperature remains constant) gives
  the Jean's mass trivially scaling as
  \begin{equation}
    \label{eq:65}
    M_{\mathrm{J}}\propto \frac{1}{\rho^{1/2}}
  \end{equation}
  This allows for the possibility of fragmentation since the mass will
  remain above the Jean's mass. More likely, this will be the case at
  earlier times, and then later the collapse becomes more adiabatic
  until a freeze-out. What ends the fragmentation is that at high
  density, the optical depths are increasing and the gas cannot
  radiate on the contraction timescale. If we can analyze the
  microphysics governing these processes, we could determine a minimum
  Jean's mass, which would explain the fragmentation. At low metal
  content, the minimum masses are much higher. This is thought to be
  the reason why the first stars were so large.\\

  \n When cooling is efficient, then the collapse and fragmentation
  occurs on the dynamical timescale:
  \begin{equation}
    \label{eq:66}
    t_{\mathrm{dyn}}\approx \frac{1}{\sqrt{G\rho}}=\frac{10^7\
      \mathrm{years}}{(n/100\ \mathrm{cm^{-3}})^{1/2}}
  \end{equation}
  Then it typically takes $1-10\ \mathrm{Myrs}$ for the collapse to
  occur. At the center, an object at hydrostatic balance forms while
  the outer layers are still collapsing (the collapse is not
  homologous). This causes an \textbf{accretion luminosity} of
  \begin{equation}
    \label{eq:67}
    L\approx \dot{M}\frac{G M_c}{R_c}
  \end{equation}
  which is powered by a loss of gravitational potential energy. The
  shocks on the core surface typically have energies of
  \begin{equation}
    \label{eq:68}
    kT_{\mathrm{shock}}\approx \frac{GM_c m_p}{R_c}\approx 0.1\ \mathrm{keV}
  \end{equation}

  \subsection{Pre-Main Sequence Stars}
  \label{sec:pre-main-sequence}

  Recall the main equations of stellar structure:
  \begin{align}
    \label{eq:69}
    \frac{dP}{dr}&=-\rho(r)g(r)\\
    dm(r)&=4\pi r^2\rho(r)\,dr\\
    \nonumber\textrm{Fluxes defined}&\textrm{ by temperature gradient}
  \end{align}
  We need to be able to have an equation describing energy (the next
  moment, as it were). Looking at the second law of thermodynamics, we
  have
  \begin{equation}
    \label{eq:70}
    dQ=TdS=dE+PdV
  \end{equation}
  The corresponding time-dependent equation in the Lagrangian is
  \begin{equation}
    \label{eq:71}
    T\frac{dS}{dt}=\frac{dE}{dt}+P\frac{dV}{dt}
  \end{equation}
  In terms of the specific units (per unit mass), where now $E$ is the
  specific internal energy,
  \begin{equation}
    \label{eq:72}
    dQ=dE+Pd\left(\frac{1}{\rho}\right)=dE-\frac{P}{\rho^2}d\rho
  \end{equation}
  The Lagrangian equation becomes
  \begin{equation}
    \label{eq:73}
    T\frac{ds}{dt}=\frac{dQ}{dt}\equiv\textrm{loss or gain}
  \end{equation}
  First we have heat gained via neuclear reactions,
  $\varepsilon$. Secondly, there is heat gained or lost due to a
  gradient in $\mathbf{F}$. Then \eqref{eq:73} becomes our last equation
  for stellar evolution:
  \begin{equation}
    \label{eq:74}
    \boxed{T\frac{ds}{dt}=\varepsilon_{\mathrm{nuc}}-\frac{\bm{\nabla}\cdot\mathbf{F}}{\rho}}
  \end{equation}
  Here we have $\mathbf{F}=F_r\hat{r}$. Then
  \begin{equation}
    \label{eq:75}
    \frac{\bm{\nabla}\cdot\mathbf{F}}{\rho}=\frac{1}{\rho}\frac{1}{r^2}\frac{\partial}{\partial
      r}\left(r^2F_r\right)
  \end{equation}
  and the luminoisty is
  \begin{equation}
    \label{eq:76}
    L(r)=4\pi r^2F_r
  \end{equation}
  Then we can rewrite \eqref{eq:75} as
  \begin{equation}
    \label{eq:77}
    \frac{\bm{\nabla}\cdot\mathbf{F}}{\rho}=\frac{1}{\rho4\pi
      r^2}\frac{\partial}{\partial r}L(r)=\frac{d L(r)}{dm(r)}
  \end{equation}
  So in the case of no nuclear burning (a pre-main sequence star), we
  have
  \begin{equation}
    \label{eq:78}
    T\frac{ds}{dt}=-\frac{d L(r)}{dm(r)}
  \end{equation}
  We assume that the lumonisty gradient is positive due to the
  temperature gradient, so under the absence of an internal energy
  source, the entropy will decrease. It will decrease at the rate set
  by the heat loss to infinity. Adding on the rest to \eqref{eq:78},
  \begin{equation}
    \label{eq:79}
    T\frac{ds}{dt}=-\frac{d L(r)}{dm(r)}=\frac{dE}{dt}-\frac{P}{\rho^2}\frac{d\rho}{dt}
  \end{equation}
  Note that the internal energy is
  \begin{equation}
    \label{eq:79a}
    E=\frac{3}{2}\frac{kT}{\mu m_p}=\frac{3}{2}\frac{P}{\rho}
  \end{equation}
  Inserting this into \eqref{eq:79} gives us
  \begin{equation}
    \label{eq:5}
    T\frac{ds}{dt}=\frac{3}{2}\frac{P}{\rho}\frac{d}{dt}\ln \left( P/\rho^{5/3}\right)=-\varepsilon_{\mathrm{grav}}
  \end{equation}
  Textbooks will often refer to this as
  $-\varepsilon_{\mathrm{grav}}$. This is done so that
  \begin{equation}
    \label{eq:6}
    \frac{\partial L(r)}{\partial
      m(r)}=\varepsilon_{\mathrm{nuc}}+\varepsilon_{\mathrm{grav}}
  \end{equation}
%%%%%%%%%%%%%%%%%%%%%%%%%%%%
% Monday, February 6, 2012 %
%%%%%%%%%%%%%%%%%%%%%%%%%%%%
  \textit{Monday, February 6, 2012}\\

  \n Let's do some dimensional analysis for a contracting star. The
  pressure is
  \begin{equation}
    \label{eq:7}
    P\sim\frac{GM^2}{R^4}\sim\frac{GM^2}{M^{4/3}}\rho^{4/3}
  \end{equation}
  Then the quantity in the log in \eqref{eq:6} would be
  \begin{equation}
    \label{eq:8}
    \frac{P}{\rho^{5/3}}\approx C\frac{1}{\rho^{1/3}}
  \end{equation}
  If the star is fully convective, then $P/\rho^{5/3}$ must be
  spatially constant, and we must have
  \begin{equation}
    \label{eq:80}
    T\frac{ds}{dt}=\frac{3}{2}\frac{P}{\rho}\frac{d}{dt}\left[\ln\left(C\rho_c^{-1/3}\right)\right]=-\frac{1}{2}\frac{P}{\rho}\frac{d\ln\rho_c}{dt}
  \end{equation}
  Assuming that $\varepsilon_{\mathrm{nuc}}=0$, \eqref{eq:5} becomes
  \begin{equation}
    \label{eq:81}
    -\frac{1}{2}\frac{P}{\rho}\frac{d\ln\rho_c}{dt}=-\frac{dL(r)}{dm(r)}\quad\Rightarrow\quad \frac{dL(r)}{dm(r)}=\frac{1}{2}\frac{P}{\rho}\frac{d\ln\rho_c}{dt}
  \end{equation}
  where $\rho_c\propto M/R^3$. Now we want to get \eqref{eq:81} in
  terms of more physical variables. Assuming $M$ is constant in time,
  \begin{equation}
    \label{eq:82}
    \frac{dL(r)}{dm(r)}=\frac{1}{2}\frac{P}{\rho}\frac{d}{dt}\left[\ln\left(\frac{1}{R^3}\right)\right]\quad\Rightarrow\quad
    \frac{dL(r)}{dm(r)}=-\frac{3}{2}\frac{P}{\rho}\frac{d}{dt}\ln R
  \end{equation}
  So, assuming a fully-convective star with no mass loss, we have
  \begin{equation}
    \label{eq:83}
    \boxed{\frac{dL(r)}{dm(r)}=-\frac{3}{2}\frac{P(r)}{\rho(r)}\frac{1}{R}\frac{dR}{dt}}
  \end{equation}
  Backtracking a bit, earlier we showed that
  \begin{equation}
    \label{eq:84}
    \mathrm{entropy}\propto \frac{P}{\rho^{5/3}}\sim\frac{GM^2}{R^4}\frac{R^5}{M^{5/3}}\propto RM^{1/3}
  \end{equation}
  for a star in hydrostatic balance. An adiabatic adjustment to this
  star must require $R\propto 1/M^{1/3}$ (entropy must remain
  constant). Thus if the mass decreases, the radius must increase
  (like in white dwarfs).\\

  \n Now we want to integrate \eqref{eq:83}. Multiplying both sides
  by $dm/dr$ and integrating from the center to the surface gives
  \begin{equation}
    \label{eq:85}
    L=-\frac{3}{2}\frac{1}{R}\frac{dR}{dt}\int_0^R\frac{P(r)}{\rho(r)}\rho(r)4\pi
    r^2\,dr=-\frac{3}{2}\frac{1}{R}\frac{dR}{dt}\int_0^RP(r)4\pi r^2\,dr
  \end{equation}
  We know the last integral from the Virial theorem's relation to the
  gravitational energy. The short answer is that
  \begin{equation}
    \label{eq:86}
    \int d^3r\,P=\frac{2}{7}\frac{GM^2}{R}
  \end{equation}
  Then the luminosity is given by
  \begin{equation}
    \label{eq:87}
    L=-\frac{3}{7}\frac{GM^2}{R^2}\frac{dR}{dt}
  \end{equation}
  This is essentially Kelvin-Helmholtz contraction, but for a star
  that is fully convective, we can derive the ``real equation'' (as
  opposed to our more bogus results earlier).\\

  \n NOte that for stars that are not changing much on the
  Kelvin-Helmholtz timescale, we may neglect $T\frac{ds}{dt}$ and
  simply write
  \begin{equation}
    \label{eq:147}
    \varepsilon_{\mathrm{nuc}}=\frac{\partial L_r}{\partial m_r}
  \end{equation}
  This is a good approximation for stars on the main sequence, where
  the burning time is much longer than the Kelvin-Helmholtz time.

  \subsection{Low Mass Stars}
  \label{sec:low-mass-stars}

  Earlier we derived that as stars are on the Hayashi track,
  \begin{equation}
    \label{eq:88}
    \frac{L}{L_\odot}\approx 0.03\left(\frac{M}{M_\odot}\right)^{4/7}\left(\frac{R}{R_\odot}\right)^2
  \end{equation}
  Plugging this result in to our previous result can give us the
  radius as a function of time:
  \begin{equation}
    \label{eq:89}
    \frac{3}{7}\frac{GM^2}{R^2}\frac{dR}{dt}=-0.03\ L_\odot\left(\frac{M}{M_\odot}\right)^{4/7}\left(\frac{R}{R_\odot}\right)^2
  \end{equation}
  We can now read off scalings: $\dot{R}\propto R^4$, $1/t\propto
  R^3$, and so $R\propto t^{-1/3}$. Explicitly, the radius for a
  fully-convective star coming down the Hayashi track
  \begin{equation}
    \label{eq:90}
    \frac{R}{R_\odot}=\left(\frac{M}{M_\odot}\right)^{10/21}\left(\frac{130\ \mathrm{Myr}}{t}\right)^{1/3}
  \end{equation}
  For masses greater than about a half of a solar mass, convection
  will dominate heat transport until the luminosity reaches the level
  that was predicted by earlier opacity arguments that assumed
  radiative diffusion. The onset of a radiative core is then roughly
  when $L\lesssim L_{\mathrm{rad}}$. The time it takes for this to
  happen is
  \begin{equation}
    \label{eq:91}
    t>10^6\ \mathrm{years}\ \left(\frac{M_\odot}{M}\right)^2
  \end{equation}
  So massive stars become radiative at a very young age. For low-mass
  stars, though,
  \begin{equation}
    \label{eq:92}
    t>2.6\times 10^6\ \mathrm{years}\ \left(\frac{M_\odot}{M}\right)^{4.4}
  \end{equation}
  For very low mass stars, this time approaches the Hubble
  time. However, very low mass stars will remain convective on the
  main sequence, so nuclear reactions will likely halt the contraction
  anyway.

  \section{Nuclear Reactions in Stars}
  \label{sec:nucl-react-stars}

  The big bang only mad (by mass) 75\% protons and 25\% Helium. In
  stars, we want to fuse protons into helium and release about 7 MeV
  per baryon, or $7\times 10^18\ \mathrm{erg\ g^{-1}}$, which is
  \emph{far} higher than the thermal energy content of matter in
  stars. As it turns out, the thermal energy is too low to bring two
  protons close enough together to fuse, so tunneling must take
  place. Additionally, there is no stable nucleus with just two (or
  more) protons, so weak interactions are required to have any stable
  reactions. 

  \subsection{Liquid Drop Model}
  \label{sec:liquid-drop-model}

  The ``size'' of a nucleus is approximately
  \begin{equation}
    \label{eq:93}
    r_0=1.3\ \mathrm{fm}\ A^{1/3}
  \end{equation}
  where $A$ is the mass number ($A=N+Z$). Then the nuclear density is
  approximately
  \begin{equation}
    \label{eq:94}
    \rho=\frac{Am_p}{\frac{4\pi}{3}R^3}\approx2\times 10^{14}\ \mathrm{g\ cm^{-3}}
  \end{equation}
  There is an energy well with
  \begin{equation}
    \label{eq:95}
    E_{\mathrm{volume}}=-14\ \mathrm{MeV}\ A
  \end{equation}
  There must also be a surface term, dictating that surface particles
  are less bound than their counterparts deeper in
  \begin{equation}
    \label{eq:96}
    N_{\mathrm{surf}}=A^{2/3}\sim\frac{\mathrm{surface}}{\mathrm{volume}}
  \end{equation}
  So then we have
  \begin{equation}
    \label{eq:97}
    E_{\mathrm{surf}}=13\ \mathrm{MeV}\ A^{2/3}
  \end{equation}
  The Coulomb energy to assemble the nucleus is necessarily
  \begin{equation}
    \label{eq:98}
    E_{\mathrm{Coulomb}}\sim\frac{e^2Z^2}{r}=\frac{3}{5}\frac{e^2
      Z^2}{R}\approx 0.56\ \mathrm{MeV}\frac{Z^2}{A^{1/3}}
  \end{equation}
  Nuclei typically prefer to have roughly equal numbers of neutrons
  and protons in order to minimize degeneracy pressure. Thus, often we
  will have $N=Z$ since then the kinetic energy is lower. This gives
  rise to a ``symmetry energy'',
  \begin{equation}
    \label{eq:99}
    E_{\mathrm{sym}}=18\ \mathrm{MeV}\frac{(N-Z)^2}{A}
  \end{equation}
%%%%%%%%%%%%%%%%%%%%%%%%%%%%%%
% Wednesday, February 8, 2012%
%%%%%%%%%%%%%%%%%%%%%%%%%%%%%%
  \textit{Wednesday, February 8, 2012}
  \subsection{Tunneling through the Coulomb Barrier}
  \label{sec:tunn-thro-coul}

  In order for two positively charged nuclei to fuse, they must get
  close enough so that the strong force can overpower the Coulomb
  potential:
  \begin{equation}
    \label{eq:100}
    V(r)=\frac{Z_1Z_2 e^2}{r}=1.4\ \mathrm{MeV}\frac{Z_1Z_2}{(r/1\ \mathrm{fm})}
  \end{equation}
  At the center of the sun, where $T_c\sim 10^7\ \mathrm{K}$, the
  thermal energy is only about $kT\sim 8.6\ \mathrm{keV}\left(T/10^8\
    \mathrm{K}\right)$. Then clearly the thermal energy in the sun is
  nowhere near high enough to overcome the Coulomb barrier. If there
  was no tunneling, we would need temperatures of order 
  \begin{equation}
    \label{eq:101}
    T\sim\frac{Z_1Z_2e^2}{kR}\sim1.6\times 10^{10}\ \mathrm{K}\
    \frac{Z_1Z_2}{(R/1\ \mathrm{fm})}
  \end{equation}
  Stars pretty much never reach this temperature, so clearly tunneling
  is how fusion happens in stars.\\

  \n There are two important ingredients for calculating reaction
  rates:
  \begin{enumerate}
  \item[1.] Tunneling
  \item[2.] Maxwell-Boltzmann distribution of particle energies
  \end{enumerate}
  The liklihood of a particle being able to tunnel through the Coulomb
  barrier is proportional to its energy, which in turn is determined
  by the Maxwell-Boltzmann statistics of the ensemble of particles. To
  get a better handle on what's going on here, consider a simple
  reaction:
  \begin{equation}
    \label{eq:102}
    \mathrm{a+X\to Y+b}
  \end{equation}
  Sometimes this is written as X(a,b)Y in astrophysics (no clue
  why). We define the \textbf{cross section} as 
  \begin{equation}
    \label{eq:103}
    \sigma \equiv \frac{\mathrm{\# reactions/X\ nucleus/unit\
        time}}{\mathrm{\#incident\ particles/cm^2/unit\ time}}
  \end{equation}
  which clearly has the dimension of length squared (area). Now
  suppose the number density of a particles is given by $n_\mathrm{a}$
  and the typical relative velocity is given by
  $v_{\mathrm{rel}}$. Then the incoming flux of particles is just
  $n_\mathrm{a}v_{\mathrm{rel}}$. From this logic, we may decuce that
  \begin{equation}
    \label{eq:104}
    n_\mathrm{a}\sigma
    v_{\mathrm{rel}}=\frac{\mathrm{\#reactions}}{\mathrm{X\ nucleus\times
        unit\ time}}
  \end{equation}
  Usually we write the reaction rate in the number of reactions per
  unit time per unit volume, or
  \begin{equation}
    \label{eq:105}
    r_{\mathrm{aX}}=n_\mathrm{X}n_{\mathrm{a}}\sigma v_{\mathrm{rel}}
  \end{equation}
  Really we need to integrate over a relative velocity distribution to
  get an understanding of the actual reaction rate, so
  \begin{equation}
    \label{eq:106}
    r_{\mathrm{aX}}=\int_0^\infty dv\
    n_\mathrm{a}n_{\mathrm{X}}\sigma(v)v\\phi(v)
  \end{equation}
  where $\phi(v)$ is the probability density of a pair of particles
  having a relative velocity $v$. More often this is written as
  \begin{equation}
    \label{eq:107}
    r_{\mathrm{aX}}=n_\mathrm{a}n_{\mathrm{X}}\avg{\sigma v}
  \end{equation}
  with
  \begin{equation}
    \label{eq:108}
    \avg{\sigma v}=\int_0^\infty dv\
    \sigma(v)v\phi(v)=4\pi\left(\frac{\mu}{2\pi kT}\right)^{3/2}
int_0^\infty v^3\sigma(v)\exp\left(-\frac{\mu v^2}{2kT}\right)dv
  \end{equation}
  where $\mu$ is now the reduced mass, $\mu=m_1m_2/(m_1+m2)$.Note that
  if both reactants are the same species, \eqref{eq:105}
  must be divided by a factor of 2 since we cannot imagine having a
  ``target'' and a ``scatterer'' (this screws up relative velocities
  and other things, see Clayton for the details). Similarly, if you
  have $N$ like reactants, we must divide by $N!$ to adjust the
  reaction rate appropriately. 

  \subsubsection{Barrier Penetration}
  \label{sec:barrier-penetration}

  Recall the Schr\"odinger equation,
  \begin{equation}
    \label{eq:109}
    \left[-\frac{\hbar^2}{2\mu}\nabla^2+V\right]\psi=E\psi
  \end{equation}
  For simplicity, we will only do a one-dimensional example to model
  barrier penetration. We supose a free particle encounters a
  finite-width and finite-height barrier. Outside the barrier, then,
  \eqref{eq:109} reduces to
  \begin{equation}
    \label{eq:110}
    -\frac{\hbar^2}{2\mu}\nabla^2\psi=E\psi\quad\Rightarrow\quad
    \psi\propto e^{-\pm ikx}
  \end{equation}
  and the energy is given by $E=\hbar^2k^2/(2\mu)$. Inside the
  barrier, though, we have decaying modes:
  \begin{equation}
    \label{eq:111}
    -\frac{\hbar^2}{2\mu}\nabla^2\psi=(E_V)\psi\quad\Rightarrow\quad
    \psi\propto e^{\pm\kappa x}
  \end{equation}
  Where the ``energy'' is $E-V=-\hbar^2\kappa^2/(2\mu)$. Now we use
  this solution in solving for the actual Coulomb potential,
  \begin{equation}
    \label{eq:112}
    V(r)=\frac{Z_1Z_2 e^2}{r}
  \end{equation}
  by using the WKB approximation. In this regime, we have
  \begin{equation}
    \label{eq:113}
    \frac{\hbar^2\kappa^2}{2\mu}=\frac{Z_1Z_2 e^2}{r}-E
  \end{equation}
  Using the WKB approximation to calculate the barrier penetration,
  which assumes that the lengthscale of changes in $V(r)$ is much
  larger than the De Broglie wavelength of the particle, we get the
  probability density being
  \begin{equation}
    \label{eq:114}
    \psi^2\propto \exp\left(-2\int \kappa\ dr\right)
  \end{equation}
  where the integral is between the two classical turning points. In
  this case, there is really only one turning point,
  \begin{equation}
    \label{eq:115}
    r_c=\frac{Z_1Z_2 e^2}{E}
  \end{equation}
  Rewriting $\kappa$ in terms of the turning point,
  \begin{equation}
    \label{eq:116}
    \kappa = \left(\frac{2\mu E}{\hbar^2}\right)^{1/2}\left[\frac{r_c}{r}-1\right]^{1/2}
  \end{equation}
  Then the integral in \eqref{eq:114} becomes
  \begin{equation}
    \label{eq:117}
    \int_{r_c}^{r_{\mathrm{in}}}\kappa dr=
      \int_{r_c}^0\frac{\pi\alpha}{2}Z_1Z_2\left(\frac{2\mu c^2}{E}\right)^{1/2}
  \end{equation}
  Here we've extended the upper limit to zero to make the integral
  doable, but this is actually a pretty good approximation (see
  Clayton for the details). This gives the probability of tunneling to
  be
  \begin{equation}
    \label{eq:118}
    P_{\mathrm{tunnel}}\propto \exp\left[-\pi\alpha
      Z_1Z_2\left(\frac{2\mu c^2}{E}\right)^{1/2}\right]=\exp\left[-\sqrt{E_\mathrm{G}/E}\right]
  \end{equation}
  Where we have defined the \textbf{Gamow} energy via
  \begin{equation}
    \label{eq:119}
    E_{\mathrm{G}}\equiv \left(\pi\alpha Z_1Z_2\right)^22\mu
    c^2\approx 0.98\ \mathrm{MeV}\ Z_1^2Z_2^2\left(\frac{\mu}{m_p}\right)
  \end{equation}
  For a proton-proton interaction, $E_{\mathrm{G}}\sim 0.5\
  \mathrm{MeV}$, and for a carbon-proton interaction,
  $E_{\mathrm{G}}\sim 33\ \mathrm{MeV}$. These values are
  significantly lower than that required for straight-up thermal
  energy to do the work, but still pretty high compared to $kT_c$. The
  probabiliy for a 1 keV proton to interact with another proton is
  then
  \begin{equation}
    \label{eq:120}
    P\propto \exp\left[-\left(\frac{500\ \mathrm{keV}}{1\
          \mathrm{keV}}\right)^{1/2}\right]\sim 2\times 10^{-10}
  \end{equation}
  So we really need to be out on the Boltzmann tail for this to occur.

  \subsubsection{Nuclear Reaction Rates}
  \label{sec:nucl-react-rates}

  The De Broglie waelength of a particle is 
  \begin{equation}
    \label{eq:121}
    \lambda = \frac{h}{p}=\frac{h}{\left(kT\mu\right)^{1/2}}\approx
    10^{-10}\ \mathrm{cm}\ \textrm{in solar core}
  \end{equation}
  While the ``size'' of the nucleus is
  \begin{equation}
    \label{eq:122}
    r_{\mathrm{nuc}}\approx 1.3\times 10^{-13}\ \mathrm{cm}\ A^{1/3}
  \end{equation}
  Clearly, classical scattering just isn't going to cut it since the
  De Broglie wavelength is much longer than the target size. Partial
  wave analysis can get an effective cross-section given by
  \begin{equation}
    \label{eq:123}
    \pi\lambda^2\left(\textrm{dimensionless stuff}\right)\exp\left[-\left(\frac{E_\mathrm{G}}{E}\right)^{1/2}\right]
  \end{equation}
  Where the ``dimensionless stuff'' comes from nasty nuclear
  physics. Often this is written in terms of the reduced wavelength,
  $\lambda/2\pi$, which I'll denote as $\bar{\lambda}$ (\LaTeX\ isn't
  agreeing with ``\verb!\lambdabar!'').
  \begin{equation}
    \label{eq:124}
    \pi\lambda^2=4\pi^3\bar{\lambda}^2=\frac{4\pi^3}{k^2}
  \end{equation}
  So the energy is
  \begin{equation}
    \label{eq:125}
    E\approx \frac{\hbar^2k^2}{2\mu}\quad\Rightarrow\quad
    4\pi^3\bar{\lambda}^2=\frac{2\pi^3\hbar^2}{\mu E}=2000\
    \mathrm{barns}\ \left(\frac{\mathrm{keV}}{E}\right)
  \end{equation}
  Then the cross section can be written as
  \begin{equation}
    \label{eq:126}
    \sigma(E)=\frac{S(E)}{E}\exp\left[-\left(\frac{E_{\mathrm{G}}}{E}\right)^{1/2}\right]
  \end{equation}
  Where the ``stupid factor'', $S(E)$ is a slowly changing function of
  the energy that takes into account the details of the nuclear
  reaction. Typical values of $S(E)$ are around 2000 keV barns. One
  important exception is the Deuterium synthesis reaction:
  \begin{equation}
    \label{eq:127}
    \mathrm{p+p\to D+e^++\nu_e}
  \end{equation}
  where $S\sim 4\times 10^{-22}\ \mathrm{kev\ barns}$. This value is
  so small because a weak interaction is involved. Now we return back
  to calculating $\avg{\sigma v}$, using our newly found cross-section
  and writing the energy as $E=\frac{1}{2}\mu v^2$ and $dE=\mu v\
  dv$. (Note that the non-relativistic energies are fine for our
  purposes.) Then we have
  \begin{equation}
    \label{eq:128}
    \avg{\sigma
      v}=\frac{1}{(kT)^{3/2}}\left(\frac{8}{\pi\mu}\right)^{1/2}\int_0^\infty
    dE\ S(E)\exp\left[-\frac{E}{kT}-\left(\frac{E_\mathrm{G}}{E}\right)^{1/2}\right]
  \end{equation}
  Before evaluating this integral, we'll need to know how to use the
  method of steppest descent. Suppose we have an integral with
  \begin{equation}
    \label{eq:129}
    I=\int_{-1}^{\infty}g(x)e^{-f(x)}dx
  \end{equation}
  where $g(x)$ is slowly varying and $f(x)$ is sharply peaked. (In our
  case, the Maxwell-Boltzmann distribution multiplied by the
  probability of tunneling is sharply peaked. Particles far out on the
  Maxwell-Boltzmann tail are very likely to tunnel, but incredibly
  unlikely to exist. Likewise, particles with low energies are in
  ample supply, but they aren't tunneling anytime soon, so there is
  some magic window in the middle that is most relevant.) We Taylor
  expand $f(x)$ about the peak (where $f'(x_0)=0$):
  \begin{equation}
    \label{eq:130}
    f(x)\approx f(x_0)+\frac{1}{2}f''(x_0)(x-x_0)^2
  \end{equation}
  Then \eqref{eq:129} can be approximated by
  \begin{equation}
    \label{eq:131}
    I=g(x_0)e^{-f(x_0)}\int_{-\infty}^{\infty}\exp\left[-\frac{f''(x_0)}{2}(x_0)^2\right]=g(x_0)e^{-f(x_0)}\sqrt{\frac{2\pi}{f''(x_0)}}
  \end{equation}
  Then for our uses, we have
  \begin{equation}
    \label{eq:132}
    f(E)=\frac{E}{kT}+\left(\frac{E_{\mathrm{G}}}{E}\right)^{1/2}
  \end{equation}
  Solving for $E_0$ gives
  $E_0^3=\frac{1}{4}E_{\mathrm{G}}(kT)^2$. Then the relevant functions
  and derivatives are
  \begin{align}
    \label{eq:133}
    f(E_0)&=3\left(\frac{E_{\mathrm{G}}}{4kT}\right)^{1/3}\\
    \label{eq:133a}
    f''(E_0)&=3\left[2E_{\mathrm{G}}(kT)^5\right]^{1/6}
  \end{align}
  Using this result in \eqref{eq:128}, we get
  \begin{align}
    \label{eq:134}
    \avg{\sigma v} & \approx
    \frac{1}{(kT)^{3/2}}\left(\frac{8}{\pi\mu}\right)^{1/2}S(E_0)\exp\left[-3\left(\frac{E_{\mathrm{G}}}{4kT}\right)^{1/3}\right]\int_0^\infty
    \exp\left[-\frac{(E-E_0)^2}{(\Delta/2)^2}\right]\\
    \label{eq:134a}
    & \Rightarrow \boxed{\Delta =
      \frac{4}{2^{1/3}\sqrt{3}}E_{\mathrm{G}}^{1/5}(kT)^{5/6}}\\
    \label{eq:134b}
    &=
    \boxed{2.6\frac{E_{\mathrm{G}}^{1/6}}{\sqrt{\mu}}\frac{S(E_0)}{(kT)^{2/3}}\exp\left[-3\left(\frac{E_{\mathrm{G}}}{4kT}\right)^{1/3}\right]=\avg{\sigma v}}
  \end{align}
  Now for some numbers. For proton-proton reactions,
  $E_{\mathrm{G}}=494\ \mathrm{keV}$, $E_0=4.5\ \mathrm{keV}\
  \left(T/10^7\ \mathrm{K}\right)^{2/3}$, and $\Delta/E_0=1\cdot
  (T/10^7\ \mathrm{K})^{1/6}$. These reactions are quite prevalent in
  the solar core. However, for proton-carbon reactions,
  $E_\mathrm{G}=36\ \mathrm{MeV}$, $E_0=18\ \mathrm{keV}(T/10^7\
  \mathrm{K})^{2/3}$, and $\Delta/E_0=0.5(T/10^7\
  \mathrm{K})^{1/6}$. These do happen in the sun, but not very
  much. They are more prevalent in larger stars who derive their
  energy from the CNO cycle (more on this later).\\
%%%%%%%%%%%%%%%%%%%%%%%%%%%%%
% Monday, February 13, 2012 %
%%%%%%%%%%%%%%%%%%%%%%%%%%%%%
  
  \n \textit{Monday, February 13, 2012}\\

  \subsection{Proton-Proton Interactions}
  \label{sec:prot-prot-inter}

  \n This result makes it evident that there is a hierarchy set by the
  charges of the particles that are attempting to fuse. That is,
  elements with lower charges will fuse more easily at low
  temperatures. Once that fuel expires, the star contracts, raising
  the temperature, triggering the next set of nuclear reactions.\\

  \n The first reaction we might think would occur would be a
  proton-proton interaction. However, no such nucleus exists, so we
  instead consider Deuterium formation:
  \begin{equation}
    \label{eq:148}
    \mathrm{p+p\to D}+e^++\nu
  \end{equation}
  This requires two things:
  \begin{itemize}
  \item Fusion requires getting the two protons within a fermi of each
    other
  \item While within one fermi, the reaction needs to have a weak
    interaction:
    \begin{equation}
      \label{eq:149}
      \mathrm{p\to n+}e^++\nu
    \end{equation}
  \end{itemize}
  (Useful fact: $(m_n-m_p)c^2\approx 1.2\ \mathrm{MeV}$.) As it turns
  out, this first reaction is the rate limiting step due to the
  sluggish weak interaction. We find the nuclear power per unit mass
  via
  \begin{equation}
    \label{eq:150}
    \varepsilon_{\mathrm{nuc}}=\frac{n_1n_2\avg{\sigma v}E_{\mathrm{nuc}}}{\rho}=\frac{dL(r)}{dm(r)}
  \end{equation}
  For a proton-proton interaction,
  \begin{equation}
    \label{eq:151}
    \avg{\sigma v}_{\mathrm{pp}}=7\times 10^{-13}\left(\frac{S}{S_s}\right)\frac{1}{T_7^{2/3}}\exp\left(-\frac{15.7}{T_7^{1/3}}\right)
  \end{equation}
  Where $S_s=2000\ \mathrm{barn\ keV}$ (1 barn$=10^{-24}\
  \mathrm{cm^2}$). Then \eqref{eq:150} becomes
  \begin{equation}
    \label{eq:152}
    \varepsilon_{\mathrm{nuc}}\approx 3\times
    10^{30}\frac{\mathrm{erg}}{\mathrm{g\ s}}\left(\frac{S}{S_s}\right)\frac{\rho}{T_7^{2/3}}\exp\left(-\frac{15.7}{T_7^{1/3}}\right)
  \end{equation}
  Of importance is that this is \emph{most} dependent on the
  temperature (due to the temperature's presence in the
  exponential). The luminosity is then set by this value via
  \begin{equation}
    \label{eq:153}
    L=\int \varepsilon\,dm=L_{\mathrm{rad}}
  \end{equation}
  This \emph{defines} the main sequence. Scaling to a low-mass star
  with $M=0.1\ M_\odot$ and $R=0.1\ R_\odot$, a fully convective star
  gives the luminosity as
  \begin{equation}
    \label{eq:154}
    L=3\times
    10^{28}\frac{\mathrm{erg}}{\mathrm{s}}\left(\frac{M}{0.1\
        M_\odot}\right)^{28/51}\left(\frac{R}{0.1\ R_\odot}\right)^4
  \end{equation}
  The ``stupid factor'' for a proton-proton interaction is
  \begin{equation}
    \label{eq:155}
    S_{\mathrm{pp}}=4\times 10^{-22}\ \mathrm{keV\ barns}=2\times
    10^{-25}S_{\mathrm{Strong}}
  \end{equation}
  As a first-order estimate, suppose that \emph{all} the mass is
  available for burning, so 
  \begin{equation}
    \label{eq:156}
    L_{\mathrm{nuc}}=M\varepsilon(T_c)
  \end{equation}
  For a fully-convective star, the central temperature is given by
  \begin{equation}
    \label{eq:157}
    T_c=0.54\frac{GM\mu m_p}{k}
  \end{equation}
  Scaling to the units designated earlier, $m=M/0.1\ M_\odot$,
  $r=R/0.1\ R_\odot$
  \begin{equation}
    \label{eq:158}
    T_7=\frac{T}{10^7\ \mathrm{K}}=0.74\frac{m}{r}
  \end{equation}
  and
  \begin{equation}
    \label{eq:159}
    \rho=139 m/r^3\ \mathrm{g\ cm^{-3}}
  \end{equation}
  Now the luminosity is
  \begin{equation}
    \label{eq:160}
    L_{\mathrm{nuc}}=M\varepsilon(T_c)=2.4\times 10^{40}\ \mathrm{erg\
      s^{-1}}\ \frac{m^{4/3}}{r^{7/3}}\exp\left(-\frac{17.35
        r^{1/3}}{m^{1/3}}\right)=2.5\times 10^{28}\ \mathrm{erg\ s^{-1}}\ m^{1/2}r^4
  \end{equation}
  Solving this transcendental equation for $r$ gives
  \begin{equation}
    \label{eq:161}
    r=m\left[1.58+0.048\ln m-0.36\ln r\right]^3
  \end{equation}
  and for the $0.1\ M_\odot$ star with just proton-proton
  interactions, the central temperature is
  \begin{equation}
    \label{eq:162}
    T_c=3.4\times 10^6\ \mathrm{K}
  \end{equation}
  So the radius is ``chosen'' so that the nuclear burning is adequate
  to match the heat losses due to radiative diffusion and convection.

  \subsection{Sidebar: The Deuterium Main Sequence}
  \label{sec:deut-main-sequ}

  Before the onset of main sequence hydrogen burning, protostars will
  undergoe Deuteriume burning via
  \begin{equation}
    \label{eq:163}
    \mathrm{p+D\to {}^3He+\gamma}
  \end{equation}
  Interestingly, the enrgy released per gram of the fusion is
  comparable to the thermal energy. When D ``burns'', the energy
  relased per gram is $kT/m_p$. This energy release slows the
  contraction for a finite time until all of the Deuterium is burned
  up.

  \subsection{The PP Chain and the CN Cycle}
  \label{sec:pp-chain}

  So once protons fuse to Deuterium, what happens? There is a
  well-established chain of events given by
  \begin{align}
    \label{eq:164}
    \mathrm{p+p}&\to \mathrm{D+}e^++\nu_e\\
    \label{eq:164a}
    \mathrm{p+D} &\to\mathrm{{}^3He+\gamma}\\
    \label{eq:164b}
    \mathrm{{}^3He+{}^3He}&\to \mathrm{{}^4He+2p}
  \end{align}
  Note that the first reaction is a weak interaction whereas the
  second two are strong interactions. At the center of the sun, we'll
  say that the central temperature is around $T_c\approx 2\times 10^7\
  \mathrm{K}$. Then the luminosity due to this chain is
  \begin{equation}
    \label{eq:165}
    L_n\propto \frac{S}{S_{\mathrm{Strong}}}\exp\left(-3\left(\frac{E_\mathrm{G}}{4kT}\right)^{1/3}\right)
  \end{equation}
  where the Gamow energy is
  \begin{equation}
    \label{eq:166}
    E_{\mathrm{G}}=\left(\pi\alpha Z_1Z_2\right)^22m_rc^2
  \end{equation}
  Now let's compare the PP chain luminosity against a reaction
  involving fusing a proton to some heavier nucleus with atomic number
  $Z$ (a strong reaction):
  \begin{equation}
    \label{eq:167}
    10^{-25}\exp\left(-3\left(\frac{E_{\mathrm{g,pp}}}{4kT}\right)^{1/3}\right)=(1)\exp\left(-3\left(\frac{E_{\mathrm{G,?}}}{4kT}\right)^{1/3}\right)
  \end{equation}
  Solving this for the Gamow energy in question, we have
  \begin{equation}
    \label{eq:168}
    E_{\mathrm{G,?}}=\left[E_{\mathrm{G,pp}}^{1/3}+19.2(4kT)^{1/3}\right]^{3}\quad
    \Rightarrow \quad E_{\mathrm{G}}=\left(19+29 T_7^{1/3}\right)^{3}\ \mathrm{keV}
  \end{equation}
  This means this happens when $T\approx 10^7$, where
  $E_{\mathrm{G}}=50\ \mathrm{MeV}$. Notably, $Z=7$ when $T_7=1$. In
  general, 
  \begin{equation}
    \label{eq:169}
    \boxed{Z_c=5\left(T_7^{1/3}+0.27\right)^{3/2}}
  \end{equation}
  Essentially what's happening here is that we are trading a higher
  Coulomb barrier in exchange for a purely strong reaction. Of
  particular interest is the CN cycle:
  \begin{align}
    \label{eq:170}
    \mathrm{p+{}^{13}C} & \to \mathrm{{}^{13}N+\gamma}\\
    \label{eq:170a}
    \mathrm{{}^{13}N} & \to \mathrm{{}^{13}C+}e^++\nu_e\\
    \label{eq:170b}
    \mathrm{p+{}^{13}C} &\to \mathrm{{}^{14}N+\gamma}\\
    \label{eq:170c}
    \mathrm{p+{}^{14}N} &\to \mathrm{{}^{15}O+\gamma}\\
    \label{eq:170d}
    \mathrm{{}^{15}O} &\to\mathrm{{}^{15}N}+e^++\nu_e\\
    \label{eq:170e}
    \mathrm{p+{}^{15}N} &\to\mathrm{{}^{12}C+\alpha}
  \end{align}
  None of these reactions require a two-particle weak interaction, so
  when the temperature is high enough, this catalytic chain is
  competitive with the PP chain. It's not obvious, but the slowest
  step in this reaction is \eqref{eq:170c}. Now we already showed that
  the nuclear luminosity goes as
  \begin{equation}
    \label{eq:171}
    L_{\mathrm{nuc}}\propto
    \exp\left(-3\left(\frac{E_{\mathrm{G}}}{4kT}\right)^{1/3}\right)
  \end{equation}
  Then the logarithmic derivative with respect to $ln T$ is
  \begin{align}
    \label{eq:172}
    \frac{d\ln L_{\mathrm{nuc}}}{d\ln T}=\left(\frac{E_{\mathrm{G}}}{4kT}\right)^{1/3}
  \end{align}
  This effectively tells us the temperature sensitivity of the
  reaction rates. As it turns out, the PP chain has a lower
  temperature sensitivity than the CN cycle. As an example, the
  proton-proton interaciton has $d\ln L/d\ln T\approx 5$ while for the
  rate-determining step in the CN cycle, we have a value of about
  24. For the sun, both of these mechanisms are at work.
\end{document}

